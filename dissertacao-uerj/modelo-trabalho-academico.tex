% --------------------------------------------------------------------
% --------------------------------------------------------------------
% Modelo de Trabalho Acadêmico utilizando classe repUERJ para elaboração
% de teses, dissertação e trabalhos monográficos em geral.
%
% Este arquivo está editado na codificação de caracteres UTF-8.
%
% As referencia estão baseadas no modelo bibtex e citação em autor-data
%
% Este modelo foi criado por Dr. Luís Fernando de Oliveira.
% Professor Adjunto do Departamento de Física Aplicada e Termodinâmica
% Instituto de Física Armando Dias Tavares
% Universidade do Estado do Rio de Janeiro - UERJ
%
% A classe repUERJ.cls foi criada a partir do código original 
% disponibilizado pelo grupo CódigoLivre (coordenado por
% Gerald Weber).
% Foram feitas adequações para implementação das normas de elaboração
% de teses e dissertações da UERJ.
%
% Os estilos repUERJformat.sty codificam os elementos pré-textuais e
% pós-textuais.
% O estilo repUERJpseudocode.sty codifica a elaboração de algoritmos
% utilizando um glossário desenvolvido por mim (Luís Fernando), o mesmo
% usado em meu curso de Física Computacional.
%
% Todo este material está disponível também no meu site
%      http://sites.google.com/site/deoliveiralf
%
% As normas da UERJ para elaboração de teses e dissertações pode ser
% obtidas no documento disponível no site
%      http://www.bdtd.uerj.br/roteiro_uerj_web.pdf
%
% Agradecimentos ao grupo da Rede Sirius - BDTD e à Biblioteca Setorial
% da Física.
%
% Rio de Janeiro, 24 de Novembro de 2015.
%
% --------------------------------------------------------------------
% --------------------------------------------------------------------
%
\documentclass[a4paper,12pt,oneside,onecolumn,final,fleqn]{repUERJ}
% ---
% Pacotes fundamentais 
% ---
\usepackage[brazil]{babel}  % adequação para o português Brasil
\usepackage[utf8]{inputenc} % Determina a codificação utilizada
                            % (conversão automática dos acentos)
\usepackage{makeidx}        % Cria o índice
\usepackage{hyperref}       % Controla a formação do índice
\usepackage{indentfirst}    % Endenta o primeiro paragrafo de
                            % cada seção.
\usepackage{graphicx}       % Inclusão de gráficos
\usepackage{subfig}
\usepackage{amsmath}        % pacote matemático
% ---
% Pacote auxiliar para as normas da UERJ
% ---
% font=default/times/lmodern/sans
\usepackage[frame=no,algline=yes,font=default]{repUERJformat}
\usepackage{repUERJpseudocode}
% ---
% Pacotes de citacoes
% ---
\usepackage[alf]{abntex2cite}

% ********************************************************************
% Informações de autoria e institucionais
% ********************************************************************

%---------------------------------------------------------------------
% Imagens pré-textuais (precisam estar no mesmo diretório deste arquivo .tex)
%---------------------------------------------------------------------

\logo{logo_uerj_cinza.png}
\marcadagua{marcadagua_uerj_cinza.png}{1}{160}{255}

%---------------------------------------------------------------------
% Informações da instituição
%---------------------------------------------------------------------

\instituicao{Universidade do Estado do Rio de Janeiro}
            {Centro de Tecnologia e Ciências}
            {Instituto de Física}{Armando Dias Tavares} 
            
%---------------------------------------------------------------------
% Informações da autoria do documento
%---------------------------------------------------------------------

\autor{Nome}
      {Sobrenome}
      {Iniciais. Do. Nome.}

\titulo{Título do trabalho acadêmico}

% se não for usar a quarta palavra chave, deixar o campo vazio: {}
\palavraschaves{primeira palavra chave}
               {segunda palavra chave}
               {terceira palavra chave}
               {}

\title{Title of dissertation}
\keywords{first keyword}
         {second keyword}
         {third keyword}
         {}

\orientador{Cargo Titulação} 
           {Nome}{Sobrenome} 
           {Unidade -- Instituição} 

%coorientador é opcional
\coorientador{Cargo Titulação} 
             {Nome}{Sobrenome} 
             {Unidade -- Instituição} 
             
%---------------------------------------------------------------------
% Grau pretendido (Doutor, Mestre, Bacharel, Licenciado) e Curso
%---------------------------------------------------------------------

\grau{Doutor} % Doutor, Mestre, Bacharel, Licenciado, Graduado
\curso{Curso}
% área de concentração é opcional
%\areadeconcentracao{área}

%---------------------------------------------------------------------
% Informações adicionais (local, data e paginas)
%---------------------------------------------------------------------

\local{Rio de Janeiro} 
\data{24}{Novembro}{2015} 

% ********************************************************************
% Configurações de aparência do PDF final
% ********************************************************************
% alterando o aspecto da cor azul
\definecolor{blue}{RGB}{41,5,195}
%\definecolor{apricot}{RGB}{251,206,177}

% informações do PDF
\hypersetup{
  %backref=true,
  %pagebackref=true,
  %bookmarks=true,
  unicode=false,
  pdftitle={\UERJtitulo},
  pdfauthor={\UERJautor},
  pdfsubject={\UERJpreambulo},
  pdfkeywords={PALAVRAS}{CHAVES}{\chaveA}{\chaveB}{\chaveC}{\chaveD},
  pdfproducer={\packagename},       % producer of the document
  pdfcreator={\UERJautor},
  colorlinks=true,       % false: boxed links; true: colored links
  linkcolor=black,       % color of internal links blue
  citecolor=black,       % color of links to bibliography blue
  filecolor=black,       % color of file links magenta
  urlcolor=black,
  bookmarksdepth=4
}

% ********************************************************************
% Início do documento
% ********************************************************************
% ---
% para compilar o índice; se não for usar, comentar
% lembrar de comentar também o comando \printindex ao final do documento

\makeindex

% ********************************************************************
% ********************************************************************
\begin{document}
% ********************************************************************
% ********************************************************************

% ----------------------------------------------------------
% ELEMENTOS PRE-TEXTUAIS
% ----------------------------------------------------------

\frontmatter % marca o início dos elementos pré-textuais

% ----------------------------------------------------------
% Capa e a folha de rosto
% ----------------------------------------------------------

\capa
\folhaderosto

% ----------------------------------------------------------
% Inserir a ficha catalográfica
% ----------------------------------------------------------

% A biblioteca deverá providenciar a ficha catalográfica. Salve a ficha no formato PDF.
% Use o nome do arquivo PDF como argumento do comando. Exemplo: ficha catalográfica
% no arquivo 'ficha.pdf'

%\fichacatalografica{ficha.pdf}

% Enquanto não possuir a ficha catalográfica, use o comando sem argumentos...

\fichacatalografica{}

% ----------------------------------------------------------
% Folha de aprovação
% ----------------------------------------------------------

% membros da banca: máximo 6
\begin{folhadeaprovacao}
  \assinatura{primeiro membro titular da banca}{instituição}
  \assinatura{segundo membro titular da banca}{instituição}
  \assinatura{terceiro membro titular da banca}{instituição}
% suplente só é incluído se efetivamente substitui um titular
  \assinatura{primeiro membro suplente da banca}{instituição}
  \assinatura{segundo membro suplente da banca}{instituição}
  \assinatura{terceiro membro suplente da banca}{instituição instituição instituição instituição }
\end{folhadeaprovacao}

% ----------------------------------------------------------
% Dedicatória
% ----------------------------------------------------------

\pretextualchapter{Dedicatória}

\vfill
Texto da dedicatória

% ----------------------------------------------------------
% Agradecimentos
% ----------------------------------------------------------

\pretextualchapter{Agradecimentos}

Texto de agradecimento

% ----------------------------------------------------------
% Epigrafe (opcional)
% ----------------------------------------------------------

\pretextualchapter{}

  \vfill\
  \begin{flushright}
    Texto da epígrafe
  \end{flushright}
  
% ----------------------------------------------------------
% RESUMO
% ----------------------------------------------------------

\pretextualchapter{Resumo}

\referencia

Texto do resumo em português.\\

\imprimirchaves

% ----------------------------------------------------------
% Abstract
% ----------------------------------------------------------

\pretextualchapter{Abstract}

\reference

Abstract in English.\\

\printkeys

% ----------------------------------------------------------
% Listas de ilustrações e tabelas
% ----------------------------------------------------------

\listadefiguras
\listadetabelas

% ----------------------------------------------------------
% Outras listas
% ----------------------------------------------------------

\listadealgoritmos

% ----------------------------------------------------------
% Lista de abreviaturas e siglas
% ----------------------------------------------------------

\pretextualchapter{Lista de abreviaturas e siglas}

\abreviatura{UERJ}{Universidade do Estado do Rio de Janeiro
				  \index{UERJ@\textbf{UERJ}}}
\abreviatura{IFADT}{Instituto de Física Armando Dias Tavares}
\abreviatura{DFAT}{Departamento de Física Aplicada e Termodinâmica}
\abreviatura{ABNT}{Associação Brasileira de Normas Técnicas}
\abreviatura{ABNT}{Núcleo de Processos Técnicos}
\abreviatura{UnB}{Universidade de Brasília}
\abreviatura{CTAN}{Comprehensive TeX Archive Network}

% ----------------------------------------------------------
% Lista de simbolos
% ----------------------------------------------------------

\pretextualchapter{Lista de s\'imbolos}

\simbolo{simbolo1}{significado e/ou valor}
\simbolo{simbolo2}{significado e/ou valor}
\simbolo{simbolo3}{significado e/ou valor}

% ----------------------------------------------------------
% Sumario
% ----------------------------------------------------------

\sumario

% ----------------------------------------------------------
% ELEMENTOS TEXTUAIS
% ----------------------------------------------------------

\mainmatter % marca o início dos elementos textuais

%=====================================================================
\chapter*{Introdução}
%=====================================================================

Este modelo de trabalho acadêmico utiliza um pacote chamado \textsl{repUERJ} desenvolvido pelo Prof. Dr. Luís Fernando de Oliveira, DFAT/IFADT/UERJ, a partir do pacote \textsf{abnTeX} original, desenvolvido pelo grupo CódigoLivre, coordenado por Gerald Weber.

O método escolhido pelos desenvolvedores do \textsf{abnTeX} foi a adaptação da classe \texttt{report.cls} como base da implementação das normas da ABNT para elaboração de trabalhos acadêmicos.

A UERJ possui normas para elaboração de teses e dissertações que seguem as normas e recomendações também da ABNT. Na UERJ, a equipe responsável pela organização e apresentação das normas técnicas é o NPROTEC, núcleo este pertencente à estrutura da Rede Sirius -- Bibliotecas da UERJ.

A partir das normais dispostas pelo NPROTEC, adaptou-se o pacote original \texttt{abntex.cls}. As principais intervenções ocorreram nos elementos pré-textuais e na diagramação em geral.

O pacote \textsl{repUERJ} é composto por três arquivos: 
\begin{itemizacao}
\item \texttt{repUERJ.cls}, classe base do documento;
\item \texttt{repUERJformat.sty}, estilo desenvolvido para implementar os diferentes elementos do trabalho técnico;
\item \texttt{repUERJpseudocode.sty}, estilo desenvolvido para implementar a escrita de pseudocódigos.
\end{itemizacao}

Em relação a construção da bibliografia do trabalho acadêmico, o pacote \textsf{abnTeX} utilizava o sistema \textsf{BibTeX} com suporte para citações numeradas ou no formato autor-data. A implementação original deste pacote mostrou-se bastante adequada no que diz respeito à geração da bibliografia. Assim, o pacote \textsf{repUERJ} não dispõe uma implementação específica para bibliografia, pois utiliza os recursos fornecidos pelo \textsf{abnTeX}.

Bem, o pacote \textsf{abnTeX} foi descontinuado em 2006 e a UnB assumiu a elaboração de outro pacote, desta vez baseado no pacote \textsf{memoir}, para elaboração de trabalhos técnicos. Acolhendo a motivação do grupo original, o pacote disponibilizado pela UnB chama-se \textsf{abnTeX2}\footnote{http://www.abntex.net.br/}. Este pacote faz parte do repositório de contribuições do CTAN\footnote{http://texcatalogue.ctan.org/entries/abntex2.html}.

Entre a implementação do pacote \textsf{repUERJ} baseado no \textsf{abnTeX}
ou no \textsf{abnTeX2}, o peso recaiu sobre o pacote original pela facilidade de adaptação e manutenção. Mas o suporte para a bibliografia vem do pacote \textsf{abnTeX2}, dado que o pacote original não está mais disponível. O novo pacote mantém a estrutura de construção de referências e citações do pacote original (estilo \texttt{abntex2cite.sty}).

Assim, como pode ser visto neste modelo, a classe base para o documento é o \texttt{repUERJ.cls}. A implementação dos elementos pré-textuais, pós-textuais e outras adequações estão no estilo \texttt{repUERJformat.sty}. Este último é obrigatório. O estilo \texttt{repUERJpseudocode.sty} somente é significativo para quem desejar inserir pseudocódigos no trabalho acadêmico. O glossário e a sintaxe do pseudocódigo foram desenvolvidos por Luís Fernando seguindo o material elaborado por ele para o curso de Física Computacional\footnote{https://sites.google.com/site/deoliveiralf/disciplinas/fisica-computacional}.

Ao trabalho.\\

\begin{flushright}O autor, 2015.\end{flushright}

%=====================================================================
\chapter{Elementos pré-textuais}
%=====================================================================

Os elementos pré-textuais\index{elementos pré-textuais@\textbf{elementos pré-textuais}} são construídos a partir das informações inseridas através de comandos próprios.

Os elementos pré-textuais e seus respectivos comandos são:
\begin{itemizacao}
\item \makebox[5.5cm][l]{capa}
    \index{elementos pré-textuais@\textbf{elementos pré-textuais}!capa}
	\texttt{\textbackslash capa}
\item \makebox[5.5cm][l]{folha de rosto}
    \index{elementos pré-textuais@\textbf{elementos pré-textuais}!folha de rosto}
	\texttt{\textbackslash folhaderosto}
\item \makebox[5.5cm][l]{folha de aprovação}
    \index{elementos pré-textuais@\textbf{elementos pré-textuais}!folha de aprovação}
	\parbox[t][35pt][t]{\textwidth-8cm}{\texttt{\ignorespaces
       	\textbackslash begin\{folhadeaprovacao\}\\
	   	\textbackslash end\{folhadeaprovacao\}}}
\item \makebox[5.5cm][l]{dedicatória}
    \index{elementos pré-textuais@\textbf{elementos pré-textuais}!dedicatória}
	\texttt{\textbackslash pretextualchapter\{Dedicatória\}}
\item \makebox[5.5cm][l]{agradecimentos} 
    \index{elementos pré-textuais@\textbf{elementos pré-textuais}!agradecimentos}
	\texttt{\textbackslash pretextualchapter\{Agradecimentos\}}
\item \makebox[5.5cm][l]{epígrafe} 
    \index{elementos pré-textuais@\textbf{elementos pré-textuais}!epígrafe}
	\texttt{\textbackslash pretextualchapter\{ \}}
\item \makebox[5.5cm][l]{resumo} 
    \index{elementos pré-textuais@\textbf{elementos pré-textuais}!resumo}
	\texttt{\textbackslash pretextualchapter\{Resumo\}}
\item \makebox[5.5cm][l]{abstract} 
    \index{elementos pré-textuais@\textbf{elementos pré-textuais}!folha de aprovação}
	\texttt{\textbackslash pretextualchapter\{Abstract\}}
\item \makebox[5.5cm][l]{lista de ilustrações} 
    \index{elementos pré-textuais@\textbf{elementos pré-textuais}!lista de ilustrações}
	\texttt{\textbackslash listadefiguras};
\item \makebox[5.5cm][l]{lista de tabelas} 
    \index{elementos pré-textuais@\textbf{elementos pré-textuais}!lista de tabelas}
	\texttt{\textbackslash listadetabelas};
\item \makebox[5.5cm][l]{lista de algoritmos} 
    \index{elementos pré-textuais@\textbf{elementos pré-textuais}!lista de algoritmos}
	\texttt{\textbackslash listadealgoritmos};
\item \makebox[5.5cm][l]{lista de abreviaturas e siglas} 
    \index{elementos pré-textuais@\textbf{elementos pré-textuais}!lista de abreviaturas e siglas}
		\parbox[t][35pt][t]{\textwidth-8cm}{
        	\texttt{\textbackslash pretextualchapter\{Lista de}\\
            \texttt{abreviaturas e siglas\}}}
\item \makebox[5.5cm][l]{lista de símbolos} 
    \index{elementos pré-textuais@\textbf{elementos pré-textuais}!lista de símbolos}
	\texttt{\textbackslash pretextualchapter\{Lista de símbolos\}}
\item \makebox[5.5cm][l]{sumário} 
    \index{elementos pré-textuais@\textbf{elementos pré-textuais}!sumário}
	\texttt{\textbackslash sumario}
\end{itemizacao}

%--------------------------------------------------------
\section{Informações de autoria e institucionais}
%--------------------------------------------------------

A capa, a folha de rosto, folha de aprovação, o resumo e o abstract possuem informações que dizem respeito ao autor do trabalho, local, instituição, ao grau pretendido, curso, orientador, coorientador, título e banca examinadora\index{informações de autoria e institucionais@\textbf{Informações de autoria e institucionais}}.

Estas informações são inseridas no documento através de comandos que antecedem o início do documento propriamente dito.

Os comandos obrigatórios são:
\begin{alinea}
\item \texttt{\textbackslash autor\{nome\}\{sobrenome\}\{iniciais do nome\}}
    \index{informações de autoria e institucionais@\textbf{Informações de autoria e institucionais}!autor}
\item \texttt{\textbackslash titulo\{título do trabalho acadêmico\}}
    \index{informações de autoria e institucionais@\textbf{Informações de autoria e institucionais}!titulo}
\item \texttt{\textbackslash title\{título do trabalho acadêmico em língua estrangeira\}}
    \index{informações de autoria e institucionais@\textbf{Informações de autoria e institucionais}!title}
\item \texttt{\textbackslash palavraschaves\{primeira\}\{segunda\}\ignorespaces
									       \{terceira\}\{quarta\}}
    \index{informações de autoria e institucionais@\textbf{Informações de autoria e institucionais}!palavras chaves}
\item \texttt{\textbackslash keywords\{primeira\}\{segunda\}\ignorespaces
									 \{terceira\}\{quarta\}}
    \index{informações de autoria e institucionais@\textbf{Informações de autoria e institucionais}!keywords}
\item \texttt{\textbackslash orientador\{cargo título\}\{nome\}\ignorespaces
			                           \{sobrenome\}\{afiliação\}}
    \index{informações de autoria e institucionais@\textbf{Informações de autoria e institucionais}!orientador}
\item \texttt{\textbackslash grau\{grau pretendido\}}
    \index{informações de autoria e institucionais@\textbf{Informações de autoria e institucionais}!grau}
\item \texttt{\textbackslash curso\{curso conclu\'ido\}}
    \index{informações de autoria e institucionais@\textbf{Informações de autoria e institucionais}!curso}
\item \texttt{\textbackslash local\{cidade\}}
    \index{informações de autoria e institucionais@\textbf{Informações de autoria e institucionais}!local}
\item \texttt{\textbackslash data\{dd\}\{mês\}\{aaaa\}}
    \index{informações de autoria e institucionais@\textbf{Informações de autoria e institucionais}!data}
\end{alinea}

Os comandos facultativos são:

\begin{alinea}
\item \texttt{\textbackslash coorientador\{cargo título\}\{nome\}\ignorespaces
			                             \{sobrenome\}\{afiliação\}}
    \index{informações de autoria e institucionais@\textbf{Informações de autoria e institucionais}!coorientador}
\item \texttt{\textbackslash areadeconcentracao\{linha de pesquisa\}}
    \index{informações de autoria e institucionais@\textbf{Informações de autoria e institucionais}!área de concentração}
\end{alinea}

O parâmetro \texttt{grau pretendido} está restrito às seguintes possibilidades: \texttt{Doutor}, \texttt{Mestre}, \texttt{Bacharel}, \texttt{Licenciado} e \texttt{Graduado}.

A ficha catalográfica é fornecida pela biblioteca. Para inseri-la no corpo do documento, criou-se o comando \texttt{\textbackslash fichacatalografica\{arquivo.pdf\}}. Não há necessidade de inseri-lo enquanto o trabalho acadêmico está sendo elaborado. Depois de apresentado à biblioteca, esta deverá fornecer a ficha catalográfica correspondente devidamente preenchida no formato \textbf{pdf}. O nome do arquivo \textbf{pdf} será o argumento do comando, incluindo a extensão .pdf.

%--------------------------------------------------------
\section{Folha de aprovação}
%--------------------------------------------------------

A banca examinadora deve constar da folha de aprovação. Cada membro da banca deve encontrar uma linha devidamente identificada para que possa assinar. Esta linha e a identificação de cada membro é provida pelo comando \texttt{\textbackslash assinatura\{título nome\}\{afiliação\}}.

Como deve ter sido visto, a folha de aprovação é construída por um ambiente próprio chamado \texttt{folhadeaprovacao}. As assinaturas devem acontecer dentro deste ambiente.

\paragraph{Exemplo}

\begin{verbatim}
\begin{folhadeaprovacao}
  \assinatura{nome do membro da banca}{instituição}
  % máximo de seis assinaturas
\end{folhadeaprovacao}
\end{verbatim}

%--------------------------------------------------------
\section{Resumo em português e em língua estrangeira}
%--------------------------------------------------------

O elemento pré-textual destinado à apresentação do resumo do trabalho acadêmico, seja em língua vernácula seja em língua estrangeira, segue um padrão simples:
\begin{itemizacao}
\item título do elemento;
\item referência do trabalho;
\item corpo do resumo;
\item palavras chaves.
\end{itemizacao}

O título do elemento é inserido pelo comando \texttt{\textbackslash pretextualchapter\{título\}}. Se for o resumo em língua portuguesa, o comando será \texttt{\textbackslash pretextualchapter\{Resumo\}} e em inglês, \texttt{\textbackslash pretextualchapter\{Abstract\}}.

A referência é inserida automaticamente pelo comando \texttt{\textbackslash referencia} para o caso do resumo em português e \texttt{\textbackslash reference} para inglês.

As palavras chaves foram inseridas no início do arquivo, junto das informações de autoria e institucionais. Para que elas sejam impressas após o corpo do resumo, basta incluir o comando \texttt{\textbackslash imprimirchaves} para a versão em português e \texttt{\textbackslash printkeys} para inglês.

O corpo do texto do resumo é inserido entre a referência e as palavras chaves.

%--------------------------------------------------------
\section{Listas e sumário}
%--------------------------------------------------------

Três listas estão disponíveis no pacote \textsf{repUERJ}:
\begin{itemizacao}
\item lista de ilustrações, cujo comando é \texttt{\textbackslash listadefiguras}
\item lista de tabelas, cujo comando é \texttt{\textbackslash listadetabelas}
\item lista de algoritmos, cujo comando é \texttt{\textbackslash listadealgoritmos}
\end{itemizacao}

Duas outras listas, neste caso não numeradas, estão disponíveis:
\begin{itemizacao}
\item lista de abreviaturas e siglas;
\item lista de símbolos.
\end{itemizacao}

A forma de inseri-las difere das três listas anteriores (listas numeradas). As listas de abreviaturas, siglas e símbolos são relações de significantes e significados. Para inserir uma abreviatura ou sigla, utiliza-se o comando \texttt{\textbackslash sigla\{abreviatura ou sigla\}\{significado por extenso\}}. A inserção de um símbolo segue o mesmo padrão com o comando \texttt{\textbackslash simbolo \{símbolo\}\{significado e/ou valor\}}. Neste comando, o símbolo é interpretado com elementos matemático. Não é necessário a inserção de \$ \$, pois o comando já está em modo matemático. Neste caso, não é possível inserir símbolos com acentuação gráfica.

A lista de abreviaturas e siglas é iniciada com o comando \texttt{\textbackslash pretextualchapter \{Lista de abreviaturas e siglas\}} e seguida dos comandos \texttt{\textbackslash sigla\{abreviatura ou sigla\}\{significado por extenso\}}.

A lista de símbolos é iniciada com o comando \texttt{\textbackslash pretextualchapter\{Lista de símbolos\}} seguido dos comandos \texttt{\textbackslash simbolo\{símbolo\}\{significado e/ou valor\}}.

Por fim, o sumário é um elemento organizado automaticamente a partir das inserções de capítulos, seções, subseções, referências, anexos e apêndices.

Para inseri-lo o trabalho acadêmico, basta inserir o comando \texttt{\textbackslash sumario}.

%=====================================================================
\chapter{Elementos pós-textuais}
%=====================================================================

Os elementos pós-textuais estão presentes ao final do trabalho acadêmico, após o capítulo de conclusão. São eles:
\begin{alinea}
\item \makebox[3cm][l]{referências} 
		\texttt{\textbackslash bibliography\{arquivos bibtex\}}
\item \makebox[3cm][l]{glossário} 
		\texttt{\textbackslash pretextualchapter\{Glossário\}}
\item \makebox[3cm][l]{apêndices} 
		\texttt{\textbackslash appendix}
\item \makebox[3cm][l]{anexos} 
		\texttt{\textbackslash annex}
\item \makebox[3cm][l]{índice} 
		\texttt{\textbackslash printindex}
\end{alinea}

%--------------------------------------------------------
\section{Referências}
%--------------------------------------------------------

As referências são inseridas no trabalho através do comando \texttt{\textbackslash bibliography\{ar\-quivos bibtex\}}, onde \texttt{arquivos bibtex} é a sequência de nomes dos arquivos que contêm as seleções de referências elaboradas pelo autor e escritas segundo a sintaxe da ferramenta de formatação de referências \textsf{BibTeX} sem a extensão do arquivo (.bib).

Na ferramenta \textsf{BibTeX}, os estilos dos diferentes documentos referenciados podem ser preestabelecidos. No caso do pacote \textsf{repUERJ}, os estilos de formatação das referências está por conta da configurações já implementadas no pacote \textsf{abnTeX2}. Este provê dois sistemas de ordenamento de referências: numérica ou alfabética.

Por uma questão de escolha pessoal, este modelo de trabalho acadêmico que está sendo lido utiliza o ordenamento alfabético. Esta opção pode ser alterada no preâmbulo do documento através do comando \texttt{\textbackslash usepackage[ordenamento]\{abntex2cite\}}, onde o termo \texttt{ordenamento} deve ser substituído por \texttt{alf} para ordenação alfabética ou \texttt{num} para numerada.

Para que as normas da ABNT sejam reproduzidas corretamente, é importante que o autor do trabalho acadêmico inclua um arquivo de configuração para o pacote \texttt{abntex2cite} chamado \texttt{abnt-options4.bib} através do comando \texttt{\textbackslash citeoption\{abnt\-options4\}} ou através do comando \texttt{\textbackslash bibliography\{abnt-options4,arquivos bibtex\}}.

%--------------------------------------------------------
\section{Glossário}
%--------------------------------------------------------

O glossário é uma lista de palavras e significados similar à lista de abreviaturas e siglas. Cada termo do glossário e seu correspondente significado é inserido pelo comando \texttt{\textbackslash definicao\{termo\}\{significado\}}. A página dedicada ao glossário é identificada pelo comando \texttt{\textbackslash postextualchapter*\{Glossário\}}.

%--------------------------------------------------------
\section{Apêndices, anexos e índice}
%--------------------------------------------------------

Em linhas gerais, o \textsf{TeX} divide o documento em três partes: corpo principal de texto, apêndices e anexos.

No corpo de texto, os capítulos, seções, subseções e demais elementos estruturais são formatados com estilos próprios, como os descritos nas normas de elaboração de trabalhos acadêmicos da UERJ: título do capítulo em caixa alta e negrito precedido por numeração contínua a partir do número 1; título de seção em negrito precedido da numeração do capítulo e da seção dentro do capítulo, e assim por diante.

Os apêndices e anexos também possuem elementos estruturais. No entanto, podem ter estilos diferentes dos do corpo de texto. Pelas normas da UERJ, os estilos de apêndices e anexos são particulares. Assim, através do \textsf{LaTeX}, marca-se o fim do corpo de texto e início dos apêndices através do comando \texttt{\textbackslash appendix}. Após este comando, os estilos das seções, subseções e demais elementos são alterados para acompanhar as normas.

Da mesma forma, o fim do espaço dos apêndices e início dos anexos é marcado pelo comando \texttt{\textbackslash annex}.

Os títulos de cada apêndice e de cada anexo são inseridos pelo comando \texttt{\textbackslash postex\-tualchapter\{título\}}. Se este comando estiver no espaço dos apêndices, a página impressa correspondente apresentará o texto ``\textbf{APÊNDICE ?} -- Título'', onde `?' será substituído por uma letra. No caso de vários apêndices, a letra inicial é o `A' e avança alfabeticamente até o `Z'.

Se o comando estiver no espaço dos anexos, a única diferença do que foi descrito para apêndices é que o texto do título terá a palavra `ANEXO' em vez de `APÊNDICE'. O resto é idêntico.

Por fim, o índice remissivo de assuntos e autores é inserido via o comando \texttt{\textbackslash printindex}. Para que este tenha efeito, é necessário a inserção do comando \texttt{\textbackslash makeindex} no ínicio do documento, no preâmbulo, antes do início do documento propriamente dito.

Para que um assunto (termo) apareça no índice, é necessário a aplicação do comando \texttt{\textbackslash index\{estrutura de termo\}}.

A estrutura de termos possui uma sintaxe elaborada pelo próprio \LaTeX\ e que deve originar a diagramação correta (em árvore) dos termos. O termo principal segue o seguinte modelo de comando: \texttt{\textbackslash index\{termo principal$@$\textbackslash textbf\{Termo principal\}}

Este argumento \texttt{termo principal$@$\textbackslash textbf\{Termo principal\}} é necessário para que o mesmo seja inserido no índice com o estilo correto: termo principal em negrito, com primeira letra em caixa alta e um espaço simples separando-o do termo anterior. O primeiro subconjunto de termos dentro do termo principal é identificado pelo uso do operador `!' entre o termo principal e o termo secundário, como no exemplo: \texttt{\textbackslash index\{termo principal$@$\textbackslash textbf\{Termo principal\}!termo secundário\}}

Se houver um terceiro subconjunto de termos abaixo do termo secundário, segue-se o mesmo raciocínio, separando-se o termo terciário do secundário com o operador `!': \texttt{\textbackslash index\{termo principal$@$\textbackslash textbf\{Termo principal\}!termo secundário!termo terciário\}}

O comando \texttt{\textbackslash index} deve estar posicionado logo após o termo ao qual se refere no corpo do texto. Assim o comando será capaz de registrar a página onde o termo desejado se encontra. Pode se repetir o comando de indexação para o mesmo termo quantas vezes forem necessárias e/ou desejadas. Para cada incidência do comando de indexação, mais um número de página será acrescentado ao lado do termo no índice de assuntos.

%=====================================================================
\chapter{Figuras e tabelas}
%=====================================================================

No pacote \textsf{repUERJ}, os ambientes \texttt{figure} e \texttt{table} foram adaptados para reproduzir as normas de elaboração de teses e dissertações da UERJ.

Um corpo de figura ou tabela é composto por um título, uma legenda, fonte, figuras ou tabela. O título é inserido pelo comando tradicional \texttt{\textbackslash caption\{título\}}. A legenda é inserida pelo comando \texttt{\textbackslash legend\{texto.\}}. A fonte é inserida pelo comando \texttt{\textbackslash source\{citação\}}.

Pelas normas, tanto figura quanto tabela devem ser apresentadas centralizadas em relação à área textual da folha. Os elementos que acompanham as figuras e as tabelas devem estar alinhadas em relação à margem esquerda do corpo. 

Para tanto, incluiu-se nos ambientes \texttt{figure} e \texttt{table} mais um parâmetro: largura do corpo. Este comprimento cria uma área centralizada onde serão inseridos os elementos do corpo. No caso de figuras, sugere-se que se deixe o ajuste da largura da imagem em função do parâmetro \texttt{\textbackslash hsize}.

Como exemplos, os blocos de comandos referentes à Figura \ref{rotulo} e à Tabela \ref{mais.rotulo} são
\begin{verbatim}
\begin{figure}[!ht]{6cm}
  \caption{Título da figura.} \label{rotulo}
  \includegraphics[width=\hsize]{logo_uerj_cor.jpg}
  \legend{Texto da legenda.}
  \source{Citação da fonte ou `O autor.'.}
\end{figure}
\end{verbatim}
\begin{verbatim}
\begin{table}[!ht]{4cm}
  \caption{Título da tabela.}\label{mais.rotulo}
  \hfill\begin{tabular}{l|l}
    \hline
      X & Y\\
    \hline
      1,20 & 15,7\\
      1,23 & 15,6\\
      1,19 & 15,3\\
      1,26 & 15,1\\
      1,22 & 15,5\\
      1,16 & 15,3\\
      1,37 & 15,7\\
    \hline
  \end{tabular}\hfill
  \legend{Texto da legenda.}
  \source{Citação da fonte ou `O autor.'.}
\end{table}
\end{verbatim}

\begin{figure}[!ht]{10cm}
  \caption{Título da figura.} \label{rotulo}
  \includegraphics[width=\hsize]{logo_uerj_cor.jpg}
  \legend{Texto da legenda.}
  \source{Citação da fonte ou `O autor.'.}
\end{figure}

\begin{table}[!ht]{4cm}
  \caption{Título da tabela.}\label{mais.rotulo}
  \hfill\begin{tabular}{l|l}
    \hline
      X & Y\\
    \hline
      1,20 & 15,7\\
      1,23 & 15,6\\
      1,19 & 15,3\\
      1,26 & 15,1\\
      1,22 & 15,5\\
      1,16 & 15,3\\
      1,37 & 15,7\\
    \hline
  \end{tabular}\hfill
  \legend{Texto da legenda.}
  \source{Citação da fonte ou `O autor.'.}
\end{table}

No caso de figuras organizadas sob um mesmo título, sugere-se o uso do pacote \textsl{subfig}. A largura de cada imagem do corpo pode ser ajustada da mesma forma -- usando-se o parâmetro \texttt{\textbackslash hsize}. Veja o bloco de comandos para a Figura \ref{outro.rotulo}.%\subref{subrotulo1}.

\begin{verbatim}
\begin{figure}[!ht]{11cm}
  \caption{Título da figura.} \label{outro.rotulo}
  \subfloat[][]{
    \label{subrotulo1}
    \fbox{\includegraphics[width=0.45\hsize]{logo_uerj_cinza.png}}}\hfill
  \subfloat[][]{
    \label{subrotulo2}
    \fbox{\includegraphics[width=0.45\hsize]{marcadagua_uerj_cinza.png}}}\\
  \subfloat[][]{
    \label{subrotulo3}
    \fbox{\includegraphics[width=0.45\hsize]{logo_uerj_cor.jpg}}}\hfill
  \legend{Texto da legenda.
    Texto da imagem \subref{subrotulo1}.
    Texto da imagem \subref{subrotulo2}.
    Texto da imagem \subref{subrotulo3}.}
  \source{Citação da fonte ou `O autor'.}
\end{figure}
\end{verbatim}

\begin{figure}[!ht]{11cm}
  \caption{Título da figura.} \label{outro.rotulo}
  \subfloat[][]{
    \label{subrotulo1}
    \fbox{\includegraphics[width=0.45\hsize]{logo_uerj_cinza.png}}}\hfill
  \subfloat[][]{
    \label{subrotulo2}
    \fbox{
      \includegraphics[width=0.45\hsize]{marcadagua_uerj_cinza.png}}}\\
  \subfloat[][]{
    \label{subrotulo3}
    \fbox{
      \includegraphics[width=0.45\hsize]{logo_uerj_cor.jpg}}}\hfill
  \legend{Texto da legenda.
    Texto da imagem \subref{subrotulo1}.
    Texto da imagem \subref{subrotulo2}.
    Texto da imagem \subref{subrotulo3}.}
  \source{Citação da fonte ou `O autor'.}
\end{figure}

%=====================================================================
\chapter{Algoritmos}
%=====================================================================

Comum nas áreas de engenharia e ciências exatas, a elaboração de algoritmos em \LaTeX{} conta com vários pacotes. Com a intenção de manter compatibilidade com o pacote \textsf{repUERJ}, desenvolveu-se um pacote a parte, o \textsf{repUERJpseudocode}, que implementa a formulação de pseudocódigos segundo as regras de sintaxe definidas no curso de Física Computacional do Instituto de Física da UERJ. Neste pacote, são oferecidos dois ambientes para a elaboração de pseudocódigos:\\

\texttt{\textbackslash begin\{algorithm\}} \texttt{\textbackslash end\{algorithm\}}

\texttt{\textbackslash begin\{pseudocode\}} \texttt{\textbackslash end\{pseudocode\}}\\

O ambiente \texttt{algorithm} tem a função de apresentar o algoritmo como elemento flutuante, pelo incremento do contador de algoritmos e prover a lista de algoritmos que pode ser incluída no início do documento. Já o ambiente \texttt{pseudocode} é responsável pela inicialização dos contadores próprios das linhas e tabulações. O ambiente \texttt{pseudocode} não depende do ambiente  \texttt{algorithm}, mas o inverso não é verdadeiro.

Cada algoritmo é identificado por um título, que é inserido via comando \texttt{\textbackslash caption\{ tí\-tulo\}}. Para a correta referência no corpo de texto do documento, é facultado o uso de rótulos (\texttt{\textbackslash label\{rótulo\}}).

Veja um rápido exemplo (Algoritmo \ref{alg:rotulo}):
\begin{verbatim}
\begin{algorithm}[!ht]
    \caption{Um exemplo rápido.} \label{alg:rotulo}
    \begin{pseudocode}
    \Algoritmo{Exemplo\_1}
        \Declarar[variáveis numéricas]{$var$, $mat[3,4]$}{numéricas}
            {$var\leftarrow$0}{}
        \Declarar[variável lógica]{$resp$}{lógica}{}{}
        \Declarar[variáveis literais]{$ch$, $str[64]$}{literais}
            {$str\leftarrow$\String{sequência}}{ }
        
        \Escrever[menu de opções]{\String{Escolha uma opção:}}{}
        \Escrever{\String{a) opção 1}}{}
        \Escrever{\String{b) opção 2}}{}
        \Escrever{\String{c) opção 3}}{}
        \Escrever{\String{x) para sair}}{}
        \Comentario{força o usuário acertar uma das opções válidas}
        \Repetir
            \Ler{$opc$}{}
        \AteQue{$opc=$'a' ou $opc=$'b' ou $opc=$'c' ou $opc=$'x'}
        
        \SeEntao{$opc=$'x'}
            \Parar
        \FimSe
        \Comentario{processamento das demais opções}
        \Comentario{...........}
    \FimAlgoritmo
    \end{pseudocode}
\end{algorithm}
\end{verbatim}

\begin{algorithm}[!ht]
    \caption{Um exemplo rápido.} \label{alg:rotulo}
    \begin{pseudocode}
    \Algoritmo{Exemplo\_1}
        \Declarar[variáveis numéricas]{$var$, $mat[3,4]$}{numéricas}
            {$var\leftarrow$0}{}
        \Declarar[variável lógica]{$resp$}{lógica}{}{}
        \Declarar[variáveis literais]{$ch$, $str[64]$}{literais}
            {$str\leftarrow$\String{sequência}}{}
        
        \Escrever[menu de opções]{\String{Escolha uma opção:}}{}
        \Escrever{\String{a) opção 1}}{}
        \Escrever{\String{b) opção 2}}{}
        \Escrever{\String{c) opção 3}}{}
        \Escrever{\String{x) para sair}}{}
        \Comentario{força o usuário acertar uma das opções válidas}
        \Repetir
            \Ler{$opc$}{}
        \AteQue{$opc=$'a' ou $opc=$'b' ou $opc=$'c' ou $opc=$'x'}
        
        \SeEntao{$opc=$'x'}
            \Parar
        \FimSe
        \Comentario{processamento das demais opções}
        \Comentario{...........}
    \FimAlgoritmo
    \end{pseudocode}
\end{algorithm}

Quando o algoritmo é longo e precisa ser ``quebrado'', a retomada se dá com os mesmos ambientes, porém, marcados com ``estrela''. Isso faz com que a continuação do algoritmo não incremente o contador de algoritmos e a numeração das linhas segue normalmente, a partir da última numeração de linhas do trecho anterior.

Veja outro exemplo rápido; neste caso, Algoritmo \ref{alg:rotulo2} em duas partes:
\begin{verbatim}
\begin{algorithm}[!ht]
    \caption{O mesmo exemplo rápido.} \label{alg:rotulo2}
    \begin{pseudocode}
    \Algoritmo{Exemplo\_1}
        \Declarar[variáveis numéricas]{$var$, $mat[3,4]$}{numéricas}
            {$var\leftarrow$0}{}
        \Declarar[variável lógica]{$resp$}{lógica}{}{}
        \Declarar[variáveis literais]{$ch$, $str[64]$}{literais}
            {$str\leftarrow$\String{sequência}}{}
        
        \Escrever[menu de opções]{\String{Escolha uma opção:}}{}
        \Escrever{\String{a) opção 1}}{}
        \Escrever{\String{b) opção 2}}{}
        \Escrever{\String{c) opção 3}}{}
        \Escrever{\String{x) para sair}}{}
    \Continua
    \end{pseudocode}
\end{algorithm}
\end{verbatim}

\begin{algorithm}[!ht]
    \caption{O mesmo exemplo rápido.} \label{alg:rotulo2}
    \begin{pseudocode}
    \Algoritmo{Exemplo\_1}
        \Declarar[variáveis numéricas]{$var$, $mat[3,4]$}{numéricas}
            {$var\leftarrow$0}{}
        \Declarar[variável lógica]{$resp$}{lógica}{}{}
        \Declarar[variáveis literais]{$ch$, $str[64]$}{literais}
            {$str\leftarrow$\String{sequência}}{}
        
        \Escrever[menu de opções]{\String{Escolha uma opção:}}{}
        \Escrever{\String{a) opção 1}}{}
        \Escrever{\String{b) opção 2}}{}
        \Escrever{\String{c) opção 3}}{}
        \Escrever{\String{x) para sair}}{}
    \Continua
    \end{pseudocode}
\end{algorithm}

\begin{verbatim}
\begin{algorithm*}[!ht]
    \caption{O mesmo exemplo rápido (continuação).}
    \begin{pseudocode*}
    \Continuacao
        \Comentario{força o usuário acertar uma das opções válidas}
        \Repetir
            \Ler{$opc$}{}
        \AteQue{$opc=$'a' ou $opc=$'b' ou $opc=$'c' ou $opc=$'x'}
        
        \SeEntao{$opc=$'x'}
            \Parar
        \FimSe
        \Comentario{processamento das demais opções}
        \Comentario{...........}
    \FimAlgoritmo
    \end{pseudocode*}
\end{algorithm*}
\end{verbatim}

\begin{algorithm*}[!ht]
    \caption{O mesmo exemplo rápido (continuação).}
    \begin{pseudocode*}
    \Continuacao
        \Comentario{força o usuário acertar uma das opções válidas}
        \Repetir
            \Ler{$opc$}{}
        \AteQue{$opc=$'a' ou $opc=$'b' ou $opc=$'c' ou $opc=$'x'}
        
        \SeEntao{$opc=$'x'}
            \Parar
        \FimSe
        \Comentario{processamento das demais opções}
        \Comentario{...........}
    \FimAlgoritmo
    \end{pseudocode*}
\end{algorithm*}

Além dos ambientes, o autor pode usar os comandos \texttt{\textbackslash alglinenumbersoff} e \texttt{\textbackslash alg\-line\-numberson} para desligar/ligar a numeração das linhas do pseudocódigo.

%-------------------------------------------------------------------
\section{Elementos de documentação do algoritmo}
%-------------------------------------------------------------------

A documentação de um algoritmo é extremamente importante, pois preserva a memória do desenvolvedor em relação ao que se fez, seu propósito, objetivos, restrições, requisitos, métodos utilizados na lógica, entradas, saídas, além de registrar a autoria e identificar a lógica.

A documentação não é obrigatória no sentido da apresentação do(s) algoritmo(s). No entanto, a prática da documentação é incentivada.

Os comandos de documentação são:
\begin{alinea}
\item \texttt{\textbackslash Documentacao}\\
      \texttt{\textbackslash FimDocumentacao}
    \begin{alinea}
        \item Abrem e fecham a documentação relativa ao 
        algoritmo. O propósito é delimitar o início e o fim da
        documentação.
    \end{alinea}
    
\item \texttt{\textbackslash Título\{\textit{identificação}\}} \\
    onde:
    \begin{alinea}
        \item \texttt{\textit{identificação}}: sentença que 
        identifica a implementação que está sendo documentada.
    \end{alinea}
\item \texttt{\textbackslash Propósito\{\textit{texto}\}} \\
    onde:
    \begin{alinea}
        \item \texttt{\textit{texto}}: texto apresentando o 
        propósito da solução e os objetivos.
    \end{alinea}
\item \texttt{\textbackslash Método\{\textit{texto}\}} \\
    onde:
    \begin{alinea}
        \item \texttt{\textit{texto}}: texto apresentando o 
        método utilizado como solução do problema. Se a 
        documentação encerra vários algoritmos, os métodos 
        relativos a cada um devem estar presentes nesta 
        seção. O método \textit{per si} pode ser entendido 
        como a apresentação da lógica do algoritmo sem o
        formalismo do pseudocódigo. É mais narrativo que 
        formal.
    \end{alinea}
\item \texttt{\textbackslash Entradas\{\textit{texto}\}}\\
      \texttt{\textbackslash Saídas\{\textit{texto}\}}\\
    onde:
    \begin{alinea}
        \item \texttt{\textit{texto}}: texto apresentando as
        informações de entrada e saída do algoritmo.
    \end{alinea}
\item \texttt{\textbackslash Observações\{\textit{texto}\}} \\
    onde:
    \begin{alinea}
        \item \texttt{\textit{texto}}: texto apresentando as
        as observações pertinentes ao algoritmo, suas 
        restrições e requisitos.\\
    \end{alinea}
\end{alinea}

\noindent\textbf{Exemplo}
\begin{verbatim}
\begin{algorithm}[!ht]
    \caption{Raízes de polinômio quadrático.} \label{alg:baskara}
    \begin{pseudocode}
        \Documentacao
            \Titulo{Raízes de polinômio quadrático}
            \Proposito{
                Calcular as raízes de um polinômio quadrático 
                a partir dos coeficientes deste polinômio.}
            \Metodo{
                Um polinômio quadrático segue a expressão\\
                \inctab  $ax^2 + by + c = 0$\\
                A partir dos coeficientes a, b e c, calcula-se as
                raízes do polinômio pelo método de Baskara.
                Neste método, as raízes reais são dadas pelas 
                expressões\\
                \inctab $r_1 = (-b-\sqrt{b^2-4ac}/(2a)$\\
                \inctab $r_2 = (-b+\sqrt{b^2-4ac}/(2a)$\\
                Pela aplicação da operação de radiciação, o seu 
                argumento deve ser não negativo, pois, caso
                contrário, o resultado da operação seria um 
                valor complexo. Assim, se o termo $b^2-4ac$ é
                não negativo, o método calcula as raízes reais
                do polinômio.}
            \Entradas{
                $a$: coeficiente do termo quadrático\\
                $b$: coeficiente do termo linear\\
                $c$: termo constante}
            \Saidas{
                $r_1$ e $r_2$: raízes reais do polinômio}
            \Observacoes{
                O método só permite o cálculo das raízes reais
                do polinômio quadrático. O método não funciona 
                caso o coeficiente $a$ seja nulo, ou seja, que 
                o polinômio seja reduzido para uma expressão 
                linear.}
        \FimDocumentacao
    \end{pseudocode}
\end{algorithm}
\end{verbatim}

\begin{algorithm}[!ht]
    \caption{Raízes de polinômio quadrático.} \label{alg:baskara}
    \begin{pseudocode}
        \Documentacao
            \Titulo{Raízes de polinômio quadrático}
            \Proposito{
                Calcular as raízes de um polinômio quadrático 
                a partir dos coeficientes deste polinômio.}
            \Metodo{
                Um polinômio quadrático segue a expressão\\
                \inctab  $ax^2 + by + c = 0$\\
                A partir dos coeficientes a, b e c, calcula-se as
                raízes do polinômio pelo método de Baskara.
                Neste método, as raízes reais são dadas pelas 
                expressões\\
                \inctab $r_1 = (-b-\sqrt{b^2-4ac}/(2a)$\\
                \inctab $r_2 = (-b+\sqrt{b^2-4ac}/(2a)$\\
                Pela aplicação da operação de radiciação, o seu 
                argumento deve ser não negativo, pois, caso
                contrário, o resultado da operação seria um 
                valor complexo. Assim, se o termo $b^2-4ac$ é
                não negativo, o método calcula as raízes reais
                do polinômio.}
            \Entradas{
                $a$: coeficiente do termo quadrático\\
                $b$: coeficiente do termo linear\\
                $c$: termo constante}
            \Saidas{
                $r_1$ e $r_2$: raízes reais do polinômio}
            \Observacoes{
                O método só permite o cálculo das raízes reais
                do polinômio quadrático. O método não funciona 
                caso o coeficiente $a$ seja nulo, ou seja, que 
                o polinômio seja reduzido para uma expressão 
                linear.}
        \FimDocumentacao
    \end{pseudocode}
\end{algorithm}

%-------------------------------------------------------------------
\section{Elementos de lógica de programação}
%-------------------------------------------------------------------

Para a construção do pseudocódigo, o pacote \textsf{repUERJpseudocode} provê diversos comandos cujas sintaxes serão apresentadas a seguir.

%..............................................................
\subsection{Declaração de dados}
%..............................................................

\begin{alinea}
  \item \texttt{\textbackslash Declarar[\textit{comentário}]\{\textit{rótulos}\}\{\textit{natureza}\}\{\textit{valores inici-}\\
        \textit{ais}\}\{\textit{mecanismo de passagem}\}}\\
        onde:
        \begin{alinea}
            \item \textit{\texttt{comentário}}: texto informativo;
            \item \textit{\texttt{rótulos}}: lista de
                  identificadores;
            \item \textit{\texttt{natureza}}: numérico, literal,
                  lógico, arquivo ou definido pelo usuário via
                  definição de registro;
            \item \textit{\texttt{valores iniciais}}: inicialização
                  de dados (opcional para variáveis; obrigatório
                  para constantes);
            \item \textit{\texttt{mecanismo de passagem}}: por cópia
                  ou por referência (significativo somente para
                  elaboração de funções, subrotinas ou
                  procedimentos).
    \end{alinea}
  \item \texttt{\textbackslash Definir[\textit{comentário}]\{\textit{rótulo}\}}\\
        \texttt{\textbackslash FimRegistro}\\
        onde:
        \begin{alinea}
            \item \textit{\texttt{comentário}}: texto informativo;
            \item \textit{\texttt{rótulo}}: identificador do registro.\\
        \end{alinea}
\end{alinea}

\noindent\textbf{Exemplo}

\begin{verbatim}
\begin{pseudocode}
    \Definir{complexo}
        \Declarar[r: real; i: imaginário]{$r$, $i$}{numéricos}{}{}
    \FimRegistro
    \Definir{coord}
        \Declarar[coordenada]{$x$, $y$, $z$}{numéricos}{}{}
    \FimRegistro
    \LinhaEmBranco
    \Declarar[variáveis auxiliares]{$m$, $n$}{numéricos}{}{}
    \Declarar[variáveis auxiliares]{$i$, $j$}{numéricos}{$i\leftarrow0$}{}
    \Declarar[string]{$nome[36]$}{literal}{}{}
    \Declarar[string]{$endereco[90]$}{literal}
                     {$endereco\leftarrow$``R. Bituca''}{}
    \Declarar[gravidade ao nível do mar]{$g$}{numérico constante}
                     {$g\leftarrow9,78$}{}
    \Declarar[inicializando campos individualmente]{$p$}
        {coord}{$p.x\leftarrow0; p.y\leftarrow0;p.z\leftarrow1$}{}
    \Declarar{$q$}{coord}{$q\leftarrow(0; 0; 1)$}{}
    \Declarar[lista de 100 pontos]{$r[100]$}{coord}{}{}
    \Declarar{$z1$, $z2$}{complexos}{$z1.r\leftarrow1;
        z1.i\leftarrow0; z1.r\leftarrow0; z2.i\leftarrow1$}{}
    \Declarar[complexo com inicialização compacta]{$w$}{complexo}
        {$w\leftarrow(2;-1)$}{}
    \Declarar[arquivos de dados]{$cadastro$, $medidas$}{arquivo}{}{}
\end{pseudocode}
\end{verbatim}

\noindent\fbox{\parbox[b][274pt][t]{\textwidth}{\singlespace
\begin{pseudocode}
    \LinhaEmBranco
    \Definir{complexo}
        \Declarar[r: real; i: imaginário]{$r$, $i$}{numéricos}{}{}
    \FimRegistro
    \Definir{coord}
        \Declarar[coordenada]{$x$, $y$, $z$}{numéricos}{}{}
    \FimRegistro
    \LinhaEmBranco
    \Declarar[variáveis auxiliares]{$m$, $n$}{numéricos}{}{}
    \Declarar[variáveis auxiliares]{$i$, $j$}{numéricos}{$i\leftarrow0$}{}
    \Declarar[string]{$nome[36]$}{literal}{}{}
    \Declarar[string]{$endereco[90]$}{literal}{$endereco\leftarrow$``R. Bituca''}{}
    \Declarar[gravidade ao nível do mar]{$g$}{numérico constante}{$g\leftarrow9,78$}{}
    \Declarar[inicializando campos individualmente]{$p$}
        {coord}{$p.x\leftarrow0; p.y\leftarrow0;p.z\leftarrow1$}{}
    \Declarar{$q$}{coord}{$q\leftarrow(0; 0; 1)$}{}
    \Declarar[lista de 100 pontos]{$r[100]$}{coord}{}{}
    \Declarar{$z1$, $z2$}{complexos}{$z1.r\leftarrow1;
        z1.i\leftarrow0; z1.r\leftarrow0; z2.i\leftarrow1$}{}
    \Declarar[complexo com inicialização compacta]{$w$}{complexo}
        {$w\leftarrow(2;-1)$}{}
    \Declarar[arquivos de dados]{$cadastro$, $medidas$}{arquivo}{}{}
\end{pseudocode}
}}

%..............................................................
\subsection{Entrada e saída de dados}
%..............................................................

\begin{alinea}
\item \texttt{\textbackslash Abrir[\textit{comentário}]\{\textit{nome do arquivo }(\textit{string})\}\{\textit{rótulo}\}\{\textit{ação sobre o arquivo}\}}\\
      \texttt{\textbackslash Fechar[\textit{comentário}]\{\textit{rótulos}\}}\\
      onde:
      \begin{alinea}
          \item \textit{\texttt{comentário}}: texto informativo;
          \item \texttt{\textit{nome do arquivo}}: \textit{string}
                contendo o nome do arquivo que será aberto;
          \item \texttt{\textit{rótulo}(\textit{s})}: 
                identificador(es) declarado(s) como
                \textbf{arquivo(s)} e que estará(ão) 
                associado(s) ao(s) nome(s) do(s) arquivo(s);
          \item \texttt{\textit{ação sobre o arquivo}}: leitura
                ou escrita.
      \end{alinea}
\item \texttt{\textbackslash Ler[\textit{comentário}]\{\textit{lista de variáveis}\}\{\textit{rótulo de arquivo} (\textit{se houver})\}}\\
      onde:
      \begin{alinea}
          \item \textit{\texttt{comentário}}: texto informativo;
          \item \texttt{\textit{lista de variáveis}}: sequência
                de variáveis;
          \item \texttt{\textit{rótulo de arquivo}}:
                identificador declarado como \textbf{arquivo}
                no caso de leitura de arquivo; se não, deixar
                vazio.
      \end{alinea}
\item \texttt{\textbackslash Escrever[\textit{comentário}]\{\textit{lista de dados}\}\{\textit{rótulo de arquivo} (\textit{se houver})\}}\\
      onde:
      \begin{alinea}
          \item \textit{\texttt{comentário}}: texto informativo;
          \item \texttt{\textit{lista de dados}}: sequência de 
                dados de saída;
          \item \texttt{\textit{rótulo de arquivo}}:
                identificador declarado como \textbf{arquivo} 
                no caso de escrita em arquivo; se não, deixar
                vazio.\\
      \end{alinea}
\end{alinea}

\noindent\textbf{Exemplos}

\begin{verbatim}
\begin{pseudocode}
    \Declarar[dados]{$x, y, t$}{numéricos}{}{}
    \LinhaEmBranco
    \Ler[entrada por dispositivo padrão -- teclado]
        {$x$, $y$, $t$}{}
    \Escrever[saída em dispositivo padrão -- monitor]
        {$2*x$, $2*y$, $t*0,5$}{}
\end{pseudocode}
\end{verbatim}

\noindent\fbox{\parbox[b][69pt][t]{\textwidth}{\singlespace
\begin{pseudocode}
    \LinhaEmBranco
    \Declarar[dados]{$x, y, t$}{numéricos}{}{}
    \LinhaEmBranco
    \Ler[entrada por dispositivo padrão -- teclado]{$x$, $y$, $t$}{}
    \Escrever[saída em dispositivo padrão -- monitor]{$2*x$, $2*y$, $t*0,5$}{}
\end{pseudocode}
}}

\begin{verbatim}
\begin{pseudocode}
    \Declarar[dados]{$x, y, t$}{numéricos}{}{}
    \Declarar[arquivo 1 e arquivo 2]{$arq1$, $arq2$}{arquivos}{}{}
    \LinhaEmBranco
    \Abrir[abrindo arquivo 1]{``entrada.txt''}{$arq1$}{leitura}
    \Abrir[abrindo arquivo 2]{``saida.txt''}{$arq2$}{escrita}
    \Ler[entrada de arq1]{$x$, $y$, $t$}{$arq1$}
    \Escrever[saída em arq2]{$2*x$, $2*y$, $t*0,5$}{$arq2$}
    \Fechar[fechando arq1 e arq2]{$arq1$, $arq2$}
\end{pseudocode}
\end{verbatim}

\noindent\fbox{\parbox[b][130pt][t]{\textwidth}{\singlespace
\begin{pseudocode}
    \LinhaEmBranco
    \Declarar[dados]{$x, y, t$}{numéricos}{}{}
    \Declarar[arquivo 1 e arquivo 2]{$arq1$, $arq2$}{arquivos}{}{}
    \LinhaEmBranco
    \Abrir[abrindo arquivo 1]{``entrada.txt''}{$arq1$}{leitura}
    \Abrir[abrindo arquivo 2]{``saida.txt''}{$arq2$}{escrita}
    \Ler[entrada de arq1]{$x$, $y$, $t$}{$arq1$}
    \Escrever[saída em arq2]{$2*x$, $2*y$, $t*0,5$}{$arq2$}
    \Fechar[fechando arq1 e arq2]{$arq1$, $arq2$}
\end{pseudocode}
}}

%..............................................................
\subsection{Desvios}
%..............................................................

\begin{alinea}
  \item \texttt{\textbackslash SeEntao[\textit{comentário}]\{\textit{condição}\}}\\
        \texttt{\textbackslash SenaoSeEntao[\textit{comentário}]\{\textit{condição}\}}\\
        \texttt{\textbackslash Senao[\textit{comentário}]}\\
        \texttt{\textbackslash FimSe}\\
        onde:
        \begin{alinea}
            \item \textit{\texttt{comentário}}: texto
                  informativo;
            \item \textit{\texttt{condição}}: teste a ser
                  avaliado.\\
        \end{alinea}
\end{alinea}

\noindent\textbf{Exemplo}

\begin{verbatim}
\begin{pseudocode}
    \alglinenumbersoff
    \SeEntao[comentário 1]{$opc=0$}
        \Escrever{``Opção 0''}{}
    \SenaoSeEntao[comentário 2]{$opc=1$}
        \Escrever{``Opção 1''}{}
    \SenaoSeEntao[comentário 3]{$opc=2$}
        \Escrever{``Opção 2''}{}
    \Senao[comentário 4]
        \Escrever{``Nenhuma das opções anteriores.''}{}
    \FimSe
\end{pseudocode}
\end{verbatim}

\noindent\fbox{\parbox[b][143pt][t]{\textwidth}{\singlespace
\begin{pseudocode}
    \alglinenumbersoff
    \LinhaEmBranco
    \SeEntao[comentário 1]{$opc=0$}
        \Escrever{``Opção 0''}{}
    \SenaoSeEntao[comentário 2]{$opc=1$}
        \Escrever{``Opção 1''}{}
    \SenaoSeEntao[comentário 3]{$opc=2$}
        \Escrever{``Opção 2''}{}
    \Senao[comentário 4]
        \Escrever{``Nenhuma das opções anteriores.''}{}
    \FimSe
\end{pseudocode}
}}

%..............................................................
\subsection{Repetições}
%..............................................................

\begin{alinea}
  \item \texttt{\textbackslash Enquanto[\textit{comentário}]\{\textit{condição}\}}\\
        \texttt{\textbackslash FimEnquanto}
  \item \texttt{\textbackslash Fazer[\textit{comentário}]}\\
        \texttt{\textbackslash Enquanto[\textit{comentário}]\{\textit{condição}\}}
  \item \texttt{\textbackslash Repetir[\textit{comentário}]}\\
        \texttt{\textbackslash AteQue[\textit{comentário}]\{\textit{condição}\}}\\
        onde:
        \begin{alinea}
            \item \textit{\texttt{comentário}}: texto
                  informativo;
            \item \textit{\texttt{condição}}: teste a ser
                  avaliado.
        \end{alinea}
  \item \texttt{\textbackslash ParaDeAtePasso [\textit{comentário}]\{\textit{rótulo}\}\{\textit{valor inicial}\}\{\textit{valor \\final}\}\{\textit{razão}\}}\\
        \texttt{\textbackslash FimPara}\\
        onde:
        \begin{alinea}
            \item \textit{\texttt{comentário}}: texto
                  informativo;
            \item \textit{\texttt{rótulo}}: identificador da
                  variável contadora;
            \item \textit{\texttt{valor inicial}}: 
                  valor numérico inicial atribuído ao rótulo;
            \item \textit{\texttt{valor final}}: 
                  valor numérico final atribuído ao rótulo;
            \item \textit{\texttt{razão}}: 
                  passo incremental de progressão dos valores
                  (opcional; se não for usado, deixar vazio).\\
        \end{alinea}
\end{alinea}

\noindent\textbf{Exemplos}

\begin{verbatim}
\begin{pseudocode}
    \alglinenumbersoff
    \Enquanto[comentário 1]{$opc<0$}
        \Escrever{``Opção inválida''}{}
        \Ler{$opc$}{}
    \FimEnquanto
    \LinhaEmBranco
    \Fazer[comentário 2]
        \Escrever{``Opções de 1 a 10''}{}
        \Ler{$opc$}{}
    \Enquanto[comentário 3]{$opc<1$ ou $opc>10$}
    \LinhaEmBranco
    \Repetir[comentário 4]
        \Escrever{``Opções de 1 a 10''}{}
        \Ler{$opc$}{}
    \AteQue[comentário 5]{$opc>=1$ e $opc<=10$}
    \LinhaEmBranco
    \ParaDeAtePasso[comentário 6]{$opc$}{1}{10}{2}
        \Escrever{``opc = '', $opc$}{}
    \FimPara
\end{pseudocode}
\end{verbatim}

\noindent\fbox{\parbox[b][275pt][t]{\textwidth}{\singlespace
\begin{pseudocode}
    \alglinenumbersoff
    \LinhaEmBranco
    \Enquanto[comentário 1]{$opc<0$}
        \Escrever{``Opção inválida''}{}
        \Ler{$opc$}{}
    \FimEnquanto
    \LinhaEmBranco
    \Fazer[comentário 2]
        \Escrever{``Opções de 1 a 10''}{}
        \Ler{$opc$}{}
    \Enquanto[comentário 3]{$opc<1$ ou $opc>10$}
    \LinhaEmBranco
    \Repetir[comentário 4]
        \Escrever{``Opções de 1 a 10''}{}
        \Ler{$opc$}{}
    \AteQue[comentário 5]{$opc>=1$ e $opc<=10$}
    \LinhaEmBranco
    \ParaDeAtePasso[comentário 6]{$opc$}{1}{10}{2}
        \Escrever{``opc = '', $opc$}{}
    \FimPara
\end{pseudocode}
}}

%..............................................................
\subsection{Módulos}
%..............................................................

\begin{alinea}
  \item \texttt{\textbackslash Funcao\{\textit{identificador}\}\{\textit{argumentos}\}}\\
        \texttt{\textbackslash FimFuncao}
  \item \texttt{\textbackslash Subrotina\{\textit{identificador}\}\{\textit{argumentos}\}}\\
        \texttt{\textbackslash FimSubrotina}
  \item \texttt{\textbackslash Procedimento\{\textit{identificador}\}\{\textit{argumentos}\}}\\
        \texttt{\textbackslash FimProcedimento}\\
        onde:
        \begin{alinea}
            \item \textit{\texttt{identificador}}: nome pelo qual 
                  a função será invocada;
            \item \textit{\texttt{argumentos}}: lista de variáveis
                  usadas como passagem de informação.\\
        \end{alinea}
\end{alinea}

\noindent\textbf{Exemplos}

\begin{verbatim}
\begin{pseudocode}
    \Funcao{Maior\_de\_3}{$a$, $b$, $c$}
        \Declarar[declarações dos argumentos]{$a$, $b$, $c$}
            {numéricos}{}{por cópia}
        \LinhaEmBranco
        \SeEntao[instruções]{$a>b$ e $a>c$}
            \Retornar{$a$}
        \SenaoSeEntao[comentário 2]{$b>a$ e $b>c$}
            \Retornar{$b$}
        \Senao[comentário 4]
            \Retornar{$c$}
        \FimSe
    \FimFuncao
\end{pseudocode}
\end{verbatim}

\noindent\fbox{\parbox[b][171pt][t]{\textwidth}{\singlespace
\begin{pseudocode}
%    \alglinenumbersoff
    \LinhaEmBranco
    \Funcao{Maior\_de\_3}{$a$, $b$, $c$}
        \Declarar[declarações dos argumentos]{$a$, $b$, $c$}
            {numéricos}{}{por cópia}
        \LinhaEmBranco
        \SeEntao[instruções]{$a>b$ e $a>c$}
            \Retornar{$a$}
        \SenaoSeEntao[comentário 2]{$b>a$ e $b>c$}
            \Retornar{$b$}
        \Senao[comentário 4]
            \Retornar{$c$}
        \FimSe
    \FimFuncao
\end{pseudocode}
}}

\begin{verbatim}
\begin{pseudocode}
    \Subrotina{Maior\_de\_3}{$a$, $b$, $c$, $m$}
        \Declarar[declarações dos argumentos]{$a$, $b$, $c$}
            {numéricos}{}{por cópia}
        \Declarar{$m$}{numérico}{}{por referência}
        \LinhaEmBranco
        \SeEntao[instruções]{$a>b$ {e} $a>c$}
            \Ins{$m\leftarrow a$}
        \SenaoSeEntao{$b>a$ e $b>c$}
            \Ins{$m\leftarrow b$}
        \Senao
            \Ins{$m\leftarrow c$}
        \FimSe
    \FimSubrotina
\end{pseudocode}
\end{verbatim}

\noindent\fbox{\parbox[b][187pt][t]{\textwidth}{\singlespace
\begin{pseudocode}
%    \alglinenumbersoff
    \LinhaEmBranco
    \Subrotina{Maior\_de\_3}{$a$, $b$, $c$, $m$}
        \Declarar[declarações dos argumentos]{$a$, $b$, $c$}{numéricos}{}
                                             {por cópia}
        \Declarar{$m$}{numérico}{}{por referência}
        \LinhaEmBranco
        \SeEntao[instruções]{$a>b$ {e} $a>c$}
            \Ins{$m\leftarrow a$}
        \SenaoSeEntao{$b>a$ e $b>c$}
            \Ins{$m\leftarrow b$}
        \Senao
            \Ins{$m\leftarrow c$}
        \FimSe
    \FimSubrotina
\end{pseudocode}
}}

\begin{verbatim}
\begin{pseudocode}
    \Procedimento{Maior\_de\_3}{$a$, $b$, $c$, $m$}
        \Declarar[declarações dos argumentos]{$a$, $b$, $c$}
            {numéricos}{}{por cópia}
        \Declarar{$m$}{numérico}{}{por referência}
        \LinhaEmBranco
        \SeEntao[instruções]{$a>b$ e $a>c$}
            \Ins{$m\leftarrow a$}
        \SenaoSeEntao{$b>a$ e $b>c$}
            \Ins{$m\leftarrow b$}
        \Senao
            \Ins{$m\leftarrow c$}
        \FimSe
    \FimProcedimento
\end{pseudocode}
\end{verbatim}

\noindent\fbox{\parbox[b][187pt][t]{\textwidth}{\singlespace
\begin{pseudocode}
%    \alglinenumbersoff
    \LinhaEmBranco
    \Procedimento{Maior\_de\_3}{$a$, $b$, $c$, $m$}
        \Declarar[declarações dos argumentos]{$a$, $b$, $c$}
            {numéricos}{}{por cópia}
        \Declarar{$m$}{numérico}{}{por referência}
        \LinhaEmBranco
        \SeEntao[instruções]{$a>b$ e $a>c$}
            \Ins{$m\leftarrow a$}
        \SenaoSeEntao{$b>a$ e $b>c$}
            \Ins{$m\leftarrow b$}
        \Senao
            \Ins{$m\leftarrow c$}
        \FimSe
    \FimProcedimento
\end{pseudocode}
}}



~ \newline

\par Texto. Texto, texto Algoritmo \ref{alg:baskara}. Texto. 

\begin{algorithm}[!ht]
    \caption{Raízes de polinômio quadrático.} \label{alg:baskara}
    \begin{pseudocode}
        \Documentacao
            \Titulo{Raízes de polinômio quadrático}
            \Proposito{
                Calcular as raízes de um polinômio quadrático 
                a partir dos coeficientes deste polinômio.}
            \Metodo{
                Um polinômio quadrático segue a expressão\\
                \inctab  $ax^2 + by + c = 0$\\
                A partir dos coeficientes a, b e c, calcula-se as
                raízes do polinômio pelo método de Baskara.
                Neste método, as raízes reais são dadas pelas 
                expressões\\
                \inctab $r_1 = (-b-\sqrt{b^2-4ac}/(2a)$\\
                \inctab $r_2 = (-b+\sqrt{b^2-4ac}/(2a)$\\
                Pela aplicação da operação de radiciação, o seu 
                argumento deve ser não negativo, pois, caso
                contrário, o resultado da operação seria um 
                valor complexo. Assim, se o termo $b^2-4ac$ é
                não negativo, o método calcula as raízes reais
                do polinômio.}
            \Entradas{
                $a$: coeficiente do termo quadrático\\
                $b$: coeficiente do termo linear\\
                $c$: termo constante}
            \Saidas{
                $r_1$ e $r_2$: raízes reais do polinômio}
        \Continua
    \end{pseudocode}
\end{algorithm}

\begin{algorithm*}[!ht]
    \caption{Título do algoritmo. (continuação)}
    \begin{pseudocode*}
        \LinhaEmBranco
        \Continuacao
          \LinhaEmBranco
            \Observacoes{
                O método só permite o cálculo das raízes reais
                do polinômio quadrático. O método não funciona 
                caso o coeficiente $a$ seja nulo, ou seja, que 
                o polinômio seja reduzido para uma expressão 
                linear.}
        \Algoritmo{Raizes\_Reais}
          \Declarar{$a, m, i$}{numéricos}{}{}
          \Declarar{$n0, n$}{numéricos}{}{}
          \Ins{$m \leftarrow 13$}
          \Ins{$n0 \leftarrow 1$}
          \Comentario{repetição}
          \ParaDeAtePasso[para cada possível valor de `a']{$a$}{2}{$m-1$}{}
            \Escrever{``a = '', $a$, ``: n = \{''}{}
            \Ins[reinicia a geração com a semente n0]{$n \leftarrow n0$}
            \Comentario{nova repetição}
            \ParaDeAtePasso{$i$}{0}{$m-1$}{}
              \Ins[gerador de números aleatórios]{$n \leftarrow resto(a*n, m)$}
              \SeEntao[se fim da sequência ...]{$n == n0$}
                \Escrever{$n$,``\}''}{}
                \Parar
              \Senao
                \Escrever{$n$}{}
              \FimSe
            \FimPara
          \FimPara
    \alglinenumbersoff
          \Ins{$a \leftarrow 1$}
          \Enquanto[comentário]{$a<10$}
            \Escrever{$a$}{}
            \Ins{$a \leftarrow a+1$}
          \FimEnquanto
          \Ins{$a \leftarrow 1$}
          \Repetir[comentário]
            \Escrever{$a$}{}
            \Ins{$a \leftarrow a+1$}
          \AteQue{$a\ge10$}
    \alglinenumberson
          \Ins{$a \leftarrow 1$}
          \Fazer[comentário]
            \Escrever{$a$}{}
            \Ins{$a \leftarrow a+1$}
          \Enquanto{$a<10$}
        \FimAlgoritmo
      \FimDocumentacao
    \end{pseudocode*}
\end{algorithm*}

%=====================================================================
\chapter{Citações}
%=====================================================================

O pacote \textsf{repUERJ} adota preferencialmente o sistema de chamada nomeada autor-data para citações. Este é um dos sistemas permitidos pelas normas da UERJ. O fato deste texto mostrar como construir as citações neste sistema não exclui o uso do sistema numerado se esta for a decisão do autor.

O sistema autor-data tem seu efeito a partir da inclusão do pacote \textsf{abntex2cite} no preâmbulo do documento \textbf{tex} com a opção \texttt{[alf]}: \texttt{\textbackslash usepackage[alf]\{abntex2cite\}}.

%-------------------------------------------------------------------
\section{Citação direta e indireta -- referências explicitas}
%-------------------------------------------------------------------

As referências explicitas são aquelas onde o texto se menciona diretamente o(s) autor(es) da referência. A forma de inserção se dá pelos comandos \texttt{\textbackslash citeonline[p.\~{}n]\{re\-fe\-rên\-cia\}} e \texttt{\textbackslash cite[p.\~{}n]\{referência\}}, onde \texttt{n} é o número da página, fonte da citação. O número da página é opcional na citação.\\

\noindent\textbf{Exemplos}\\

\noindent\makebox[8cm][l]{\texttt{\textbackslash citeonline\{bib:Amado1991\}}} produz: \citeonline{bib:Amado1991}

\noindent\makebox[8cm][l]{\texttt{\textbackslash citeonline[p.\~{}101]\{bib:Amado1991\}}} produz: \citeonline[p.~101]{bib:Amado1991}

\noindent\makebox[8cm][l]{\texttt{\textbackslash cite\{bib:Amado1991\}}} produz: \cite{bib:Amado1991}

\noindent\makebox[8cm][l]{\texttt{\textbackslash cite[p.\~{}320]\{bib:Amado1991\}}} produz: \cite[p.~320]{bib:Amado1991}


%-------------------------------------------------------------------
\section{Citação de citação -- referência citada por outra referência}
%-------------------------------------------------------------------

A citação de citação aplica a expressão latina \textsl{apud}, que quer dizer ``citado por''. Os comandos que inserem citações de citações são \texttt{\textbackslash apudonline\{referência\}\{citada por\}} e \texttt{\textbackslash apud\{referência\}\{citada por\}}.\\

\noindent\textbf{Exemplos}\\

\noindent\texttt{\textbackslash apudonline\{bib:Austin2001\}\{bib:Schlemm2004\}}\\
 \hspace*{\parindent}produz: \apudonline{bib:Austin2001}{bib:Schlemm2004}

\noindent\texttt{\textbackslash apudonline[p.\~{}101]\{bib:Austin2001\}\{bib:Schlemm2004\}}\\
\hspace*{\parindent}produz: \apudonline[p.~101]{bib:Austin2001}{bib:Schlemm2004}

\noindent\texttt{\textbackslash apud\{bib:Austin2001\}\{bib:Schlemm2004\}}\\
\hspace*{\parindent}produz: \apud{bib:Austin2001}{bib:Schlemm2004}

\noindent\texttt{\textbackslash apud[p.\~{}320]\{bib:Austin2001\}\{bib:Schlemm2004\}}\\
\hspace*{\parindent}produz: \apud[p.~320]{bib:Austin2001}{bib:Schlemm2004}\\

O uso do termo \textsl{apud} na citação impõe que as referências também utilizem o mesmo termo. As referências devem ser combinadas no registro da primeira citação. A referência mais recente (aquela que cita a outra) deve ser registrada isoladamente.\\

\noindent\fbox{\parbox{\textwidth}{\citetext{bib:Austin2001}}}

\begin{verbatim}
@book{bib:Austin2001,
  author = {Austin, James},
  title = {Parcerias},
  subtitle = {fundamentos e benefícios para o terceiro setor},
  address = {São Paulo},
  publisher = {Futura},
  year = {2001},
  note = {{apud} \citetext{bib:Schlemm2004}}
}
\end{verbatim}

\noindent\fbox{\parbox{\textwidth}{\citetext{bib:Schlemm2004}}}

\begin{verbatim}
@misc{bib:Schlemm2004,
  author = {Schlemm, Marcos Mueller and Souza, Queila Regina},
  title = {COEP Paran{\'a} e empreendedorismo social},
  subtitle = {uma experiência de gestão do conhecimento para inovação},
  year = {2004},
  url = {http://www.coepbrasil.org.br/downloads/tese_queila.doc},
  urlaccessdate = {9 set. 2004}
}
\end{verbatim}

%-------------------------------------------------------------------
\section{Citação de partes da referência}
%-------------------------------------------------------------------

\noindent\textbf{Exemplos}\\

\noindent Autor(es) da referência fora da caixa alta: \texttt{\textbackslash citeauthoronline\{bib:Andrade1997\}} \\
\hspace*{\parindent}produz: \citeauthoronline{bib:Andrade1997}

\noindent Autor(es) da referência em caixa alta: \texttt{\textbackslash citeauthor\{bib:Andrade1997\}} \\
\hspace*{\parindent}produz: \citeauthor{bib:Andrade1997}

\noindent Ano da publicação: \texttt{\textbackslash citeyear\{bib:Andrade1997\}} \\
\hspace*{\parindent}produz: \citeyear{bib:Andrade1997}

%-------------------------------------------------------------------
\section{\textsf{BibTeX} e os modelos de referência}
%-------------------------------------------------------------------

As normas da UERJ estabelecem vários estilos diferentes para construção das referências. O \textsf{BibTeX} torna-se extremamente útil nesta tarefa, pois, através do pacote \textsf{abnTeX2}, os estilos são ajustados para cada tipo de produção. Confira no Apêndice \ref{sec:modelos} os modelos suportados pelo \textsf{BibTeX}.

Para uma simples conferência, listar-se-á os modelos previstos formalmente no \textsf{BibTeX}:
\begin{alinea}
\item [\textbf{article}] artigo de um jornal ou revista.\\
Campos obrigatórios: \textit{author}, \textit{title}, \textit{journal}, \textit{year} (autor, título, periódico, ano).\\
Campos opcionais: \textit{volume}, \textit{number}, \textit{pages}, \textit{month}, \textit{note} (volume, número, páginas, mês, anotação).

\item [\textbf{book}] livro com uma editora identificada.\\
Campos obrigatórios: \textit{author} ou \textit{editor}, \textit{title}, \textit{publisher}, \textit{year} (autor ou editor, título, editora, ano).\\
Campos opcionais: \textit{volume} ou \textit{number}, \textit{series}, \textit{address}, \textit{edition}, \textit{month}, \textit{note} (volume ou número, série, endereço, edição, mês, anotação).

\item [\textbf{booklet}] (livreto) obra impressa e encadernado, mas sem uma editora identificada ou instituição patrocinadora.\\
Campo obrigatório: \textit{title} (título).\\
Campos opcionais: \textit{author}, \textit{howpublished}\footnote{A tradução para \textit{howpublished} seria ``como foi editado''. Aqui, será interpretado como inserção de uma observação sobre o documento.}, \textit{address}, \textit{month}, \textit{year}, \textit{note} (autor, observação, endereço, mês, ano, anotação).

\item [\textbf{inbook}] parte de um livro, que pode ser um capítulo (ou seção ou qualquer outro elemento estrutural) e/ou uma série de páginas.\\
Campos obrigatórios: \textit{author} ou \textit{editor}, \textit{title}, \textit{chapter} e/ou \textit{pages}, \textit{publisher}, \textit{year} (autor ou editor, título, capítulo e/ou páginas, editora, ano).\\
Campos opcionais: \textit{volume} ou \textit{number}, \textit{series}, \textit{type}, \textit{address}, \textit{edition}, \textit{month}, \textit{note} (volume ou número, série, tipo, endereço, edição, mês, anotação).

\item [\textbf{incollection}] parte de um livro que tem o seu próprio título e/ou autoria.\\
Campos obrigatórios: \textit{author}, \textit{title}, \textit{booktitle}, \textit{publisher}, \textit{year} (autor, título da parte, título do livro, editora, ano).\\
Campos opcionais:\textit{editor}, \textit{volume} ou \textit{number}, \textit{series}, \textit{type}, \textit{chapter}, \textit{pages}, \textit{address}, \textit{edition}, \textit{month}, \textit{note} (editor, volume ou número, série, tipo, capítulo, páginas, endereço, edição, mês, anotação).

\item [\textbf{inproceedings}] artigo publicado em anais de evento.\\
Campos obrigatórios: \textit{author}, \textit{title}, \textit{booktitle}, \textit{year} (autor, título do artigo, título dos anais, ano).\\
Campos opcionais: \textit{editor}, \textit{volume} ou \textit{number}, \textit{series}, \textit{pages}, \textit{address}, \textit{month}, \textit{organization}, \textit{publisher}, \textit{note} (editor, volume ou número, série, páginas, endereço, mês, organização, editora, anotação).

\item [\textbf{manual}] documentação técnica, manual.\\
Campo obrigatório: \textit{title} (título).\\
Campos opcionais: \textit{author}, \textit{organization}, \textit{address}, \textit{edition}, \textit{month}, \textit{year}, \textit{note} (autor, organização, endereço, edição, mês, ano, anotação).

\item [\textbf{mastersthesis}] dissertação de mestrado.\\
Campos obrigatórios: \textit{author}, \textit{title}, \textit{school}, \textit{year} (autor, título, instituição acadêmica, ano).\\
Campos opcionais: \textit{type}, \textit{address}, \textit{month}, \textit{note} (tipo, endereço, mês, noanotaçãota).

\item [\textbf{misc}] documento de natureza indefinida, utilizado quando não se encaixa em nenhuma categoria existente.\\
Campos obrigatórios: nenhum.\\
Campos opcionais: \textit{author}, \textit{title}, \textit{howpublished}\footnote{Ver nota 4.}, \textit{month}, \textit{year}, \textit{note} (autor, título, observação, mês, ano, anotação).

\item [\textbf{phdthesis}] tese de doutorado.\\
Campos obrigatórios: \textit{author}, \textit{title}, \textit{school}, \textit{year} (autor, título, instituição acadêmica, ano).\\
Campos opcionais: \textit{type}, \textit{address}, \textit{month}, \textit{note} (tipo, endereço, mês, anotação).

\item [\textbf{proceedings}] anais de eventos (conferência, simpósio, congresso, etc).\\
Campos obrigatórios: \textit{title}, \textit{year} (título, ano).\\
Campos opcionais: \textit{editor}, \textit{volume} ou \textit{number}, \textit{series}, \textit{address}, \textit{month}, \textit{organization}, \textit{publisher}, \textit{note} (editor, volume ou número, série, endereço, mês, organização, editora, anotação).

\item [\textbf{techreport}] relatório técnico publicado por uma instituição, geralmente numerado dentro de uma série.\\
Campos obrigatórios: \textit{author}, \textit{title}, \textit{institution}, \textit{year} (autor, título, instituição, anos).\\
Campos opcionais: \textit{type}, \textit{number}, \textit{address}, \textit{month}, \textit{note} (tipo, número, endereço, mês, anotação).

\item [\textbf{unpublished}] documento com autoria e título, mas não formalmente divulgado.\\
Campos obrigatórios: \textit{author}, \textit{title}, \textit{note} (autor, título, anotação).\\
Campos opcionais: \textit{month}, \textit{year} (mês, ano).
\end{alinea}


%-----------------------------------------------------------------
\section{Exemplos de referências usando os modelos do \textsf{BibTeX}}
%-----------------------------------------------------------------

%-----------------------------------------------------------------
\subsection{Documentos no todo}
%-----------------------------------------------------------------

%..............................................................
\subsubsection{Eventos (congressos, conferências, seminários)}
%..............................................................

\noindent\fbox{\parbox{\textwidth}{\citetext{bib:proceedings}}}

\begin{verbatim}
@proceedings{bib:proceedings,
      organization = {Nome do evento},
      org-short = {Abreviatura do evento},
      conference-number = {Número},
      conference-year = {Ano da realização},
      conference-location = {Local da realização},
      title = {Título do documento},
      address = {Local da publicação},
      publisher = {Editora},
      year = {Data da publicação},
      note = {Número de páginas ou volumes}
}
\end{verbatim}

\noindent\textbf{Exemplos}\\

\noindent\fbox{\parbox{\textwidth}{\citetext{bib:CBA1996}}}

\begin{verbatim}
@proceedings{bib:CBA1996,
  organization = {Congresso Brasileiro de Automática},
  org-short = {CBA'96},
  conference-number = {11},
  conference-year = {1996},
  conference-location = {São Paulo},
  title = {Anais...},
  address = {São Paulo},
  publisher = {SBA},
  year = {1996},
  note = {3 v},
}
\end{verbatim}

\noindent\fbox{\parbox{\textwidth}{\citetext{bib:CBAS1980}}}

\begin{verbatim}
@proceedings{bib:CBAS1980,
  organization = {Congresso Brasileiro de Águas Subterrâneas},
  org-short = {CBAS'80},
  conference-number = {1},
  conference-year = {1980},
  conference-location = {Recife},
  title = {Anais...},
  address = {Recife},
  publisher = {ABAS},
  year = {1980},
  pages = {626}
}
\end{verbatim}

%..............................................................
\subsubsection{Livros, folhetos, manuais, guias, catálogos, enciclopédias, dicionários}
%..............................................................

\noindent\fbox{\parbox{\textwidth}{\citetext{bib:livro}}}

\begin{verbatim}
@book{bib:livro,
    author = {Sobrenome1, Nome1 and Sobrenome2, Nome2 and
              Sobrenome3, Nome3},
    title = {Título do livro},
    subtitle = {Subtítulo do livro},
    edition = {Edição},
    address = {Local de publicação},
    publisher = {Editora},
    year = {Data da publicação},
    pages = {Número de paginas ou volumes (opcional)}
}
\end{verbatim}%\footnote{Número de páginas ou volumes é opcional.}

\noindent\textbf{Exemplos}\\

\noindent\fbox{\parbox{\textwidth}{\citetext{bib:Borheim1976}}}

\begin{verbatim}
@book{bib:Borheim1976,
  author = {Borheim, Gerd A},
  title = {Introdu{\c c}{\~a}o ao filosofar},
  subtitle = {o pensamento filos{\'o}fico em bases existenciais},
  edition = {3},
  address = {Porto Alegre},
  publisher = {Ed. Globo},
  year = {1976},
  pages = {117},
}
\end{verbatim}

\noindent\fbox{\parbox{\textwidth}{\citetext{bib:Borheim2001}}}

\begin{verbatim}
@book{bib:Borheim2001,
  author={Borheim, Gerd A},
  title={Os fil{\'o}sofos pr{\'e}-socr{\'a}ticos},
  subtitle={o pensamento filos{\'o}fico em bases existenciais},
  edition={11},
  address={Rio de Janeiro},
  publisher={Cultrix},
  year={2001}
}
\end{verbatim}

\noindent\fbox{\parbox{\textwidth}{\citetext{bib:Rabaca2001}}}

\begin{verbatim}
@book{bib:Rabaca2001,
  author = {Raba{\c c}a, C. A. and Barbosa, G. G.},
  title = {Dicionário de Comunicação},
  edition = {Ed. rev. e atual.},
  address = {Rio de Janeiro},
  publisher = {Campus},
  year = {2001},
}
\end{verbatim}

%..............................................................
\subsubsection{Normas técnicas}
%..............................................................

\noindent\fbox{\parbox{\textwidth}{\citetext{bib:normas}}}

\begin{verbatim}
@manual{bib:normas,
    organization = {Organiza{\c c}{\~a}o respons{\'a}vel pela norma},
    org-short = {Sigla},
    title = {T{\'i}tulo da norma},
    subtitle = {Sub{\'i}tulo da norma},
    address = {Local de publicação},
    year = {data da publicação},
}
\end{verbatim}

\noindent\textbf{Exemplos}\\

\noindent\fbox{\parbox{\textwidth}{\citetext{bib:ABNT2011}}}

\begin{verbatim}
@manual{bib:ABNT2011,
    organization = {Associa{\c c}{\~a}o Brasileira de Normas T{\'e}cnicas},
    org-short = {ABNT},
    title = {NBR 14724},
    subtitle = {informação e documentação: trabalhos acadêmicos: 
                apresentação},
    address = {Rio de Janeiro},
    year = {2011},
}
\end{verbatim}

%..............................................................
\subsubsection{Publicações periódicas (revistas, boletins, anuários, etc.)}
%..............................................................

\noindent\fbox{\parbox{\textwidth}{\citetext{bib:journalpart}}}

\begin{verbatim}
@journalpart{bib:journalpart,
      title = {T\'itulo do peri\'odico},
      address = {Local da publicação},
      publisher = {Editora},
      year = {Data do primeiro volume-data do último volume},
      note = {Periodicidade}
}
\end{verbatim}

\noindent\textbf{Exemplos}\\

\noindent\fbox{\parbox{\textwidth}{\citetext{bib:AGEV1968}}}

\begin{verbatim}
@journalpart{bib:AGEV1968,
      title = {Anu{\'a}rio Internacional},
      address = {São Paulo},
      publisher = {AGEV},
      year = {1997--1978}
}
\end{verbatim}

\noindent\fbox{\parbox{\textwidth}{\citetext{bib:Bibliophilos1968}}}

\begin{verbatim}
@journalpart{bib:Bibliophilos1968,
      title = {Boletim da Sociedade de Bibliophilos Barbosa Machado},
      address = {Lisboa},
      publisher = {Impr. Libano da Silva},
      year = {1910--{ }{ }{ }{ }},
      note = {Irregular}
}
\end{verbatim}

\noindent\fbox{\parbox{\textwidth}{\citetext{bib:RevistaRio1985}}}

\begin{verbatim}
@journalpart{bib:RevistaRio1985,
      title = {Revista Rio de Janeiro},
      address = {Niterói},
      publisher = {EDUFF},
      year = {1985-{ }{ }{ }{ }},
      note = {Quadrimestral}
}
\end{verbatim}

%..............................................................
\subsubsection{Relatórios técnicos}
%..............................................................

\noindent\fbox{\parbox{\textwidth}{\citetext{bib:relatorio}}}

\begin{verbatim}
@book{bib:relatorio,
    author = {Sobrenome1, Nome1 and Sobrenome2, Nome2 and
              Sobrenome3, Nome3 and Sobrenome4, Nome4},
    title = {Título do relatório técnico},
    subtitle = {Subtítulo},
    edition = {Edição},
    address = {Local},
    publisher = {Editora},
    year = {Ano},
    pages = {Número de páginas},
    howpublished  = {Relat{\'o}rio t{\'e}cnico}
}
\end{verbatim}

\noindent\textbf{Exemplo}\\

\noindent\fbox{\parbox{\textwidth}{\citetext{bib:Silva1985}}}

\begin{verbatim}
@book{bib:Silva1985,
  author = {Silva, L. S.},
  title = {Manuten{\c c}{\~a}o de softwares},
  address = {Campinas},
  publisher = {UNICAMP-FEE-DCA},
  year = {1985},
  pages = {110},
  howpublished = {Relat{\'o}rio T{\'e}cnico},
}
\end{verbatim}

\noindent\fbox{\parbox{\textwidth}{\citetext{bib:CETESB1994}}}

\begin{verbatim}
@book{bib:CETESB1994,
    organization = {Companhia Estadual de Tecnologia de Saneamento 
                    B{\'a}sico e de Defesa do Meio Ambiente (SP)},
    org-short = {CETESB},
    title = {Bacia hidrográfica do Ribeirão Pinheiros},
    address = {São Paulo},
    publisher = {CETESB},
    year = {1994},
    pages = {39},
    note = {Relatório Técnico},
}
\end{verbatim}

%..............................................................
\subsubsection{Teses e dissertações}
%..............................................................

\noindent\fbox{\parbox{\textwidth}{\citetext{bib:tese}}}

\begin{verbatim}
@phdthesis{bib:tese,
  author = {Sobrenome, Nome},
  title = {Título do trabalho acadêmico},
  subtitle = {Subtítulo},
  year-presented = {Ano da conclusão},
  school = {Instituição},
  type = {Grau em Curso},
  address = {Local},
  year = {Ano na folha de aprovação},
  pages = {Número de folhas},
  pagename = {f.},
}
\end{verbatim}

\noindent\fbox{\parbox{\textwidth}{\citetext{bib:dissertacao}}}

\begin{verbatim}
@masterthesis{bib:dissertacao,
  author = {Sobrenome, Nome},
  title = {Título do trabalho acadêmico},
  subtitle = {Subtítulo},
  year-presented = {Ano da conclusão},
  school = {Instituição},
  type = {Grau em Curso},
  address = {Local},
  year = {Ano na folha de aprovação},
  pages = {Número de folhas},
  pagename = {f.},
}
\end{verbatim}

\noindent\fbox{\parbox{\textwidth}{\citetext{bib:monografia}}}

\begin{verbatim}
@monography{bib:monografia,
  author = {Sobrenome, Nome},
  title = {Título do trabalho acadêmico},
  subtitle = {Subtítulo},
  year-presented = {Ano da conclusão},
  school = {Instituição},
  type = {Grau em Curso},
  address = {Local},
  year = {Ano na folha de aprovação},
  pages = {Número de folhas},
  pagename = {f.},
}
\end{verbatim}

\noindent\fbox{\parbox{\textwidth}{\citetext{bib:TCC}}}

\begin{verbatim}
@thesis{bib:TCC,
  author = {Sobrenome, Nome},
  title = {Título do trabalho acadêmico},
  subtitle = {Subtítulo},
  year-presented = {Ano da conclusão},
  school = {Instituição},
  type = {Trabalho de Conclusão de Curso (Grau em Curso)},
  address = {Local},
  year = {Ano na folha de aprovação},
  pages = {Número de folhas},
  pagename = {f.},
}
\end{verbatim}

\noindent\textbf{Exemplos}\\

\noindent\fbox{\parbox{\textwidth}{\citetext{bib:Correa1997}}}

\begin{verbatim}
@PhDThesis{bib:Correa1997,
  author = {Corrêa, Marilena Cordeiro Dias Villela},
  title = {A tecnologia a servi{\c c}o de um sonho},
  subtitle = {um estudo sobre a reprodu{\c c}ão assistida},
  year-presented = {1997},
  school = {Instituto de Medicina Social,
            Universidade do Estado do Rio de Janeiro},
  type = {Doutorado em Saúde Coletiva},
  address = {Rio de Janeiro},
  year = {1998},
  pages = {290},
  pagename = {f.},
}
\end{verbatim}

\noindent\fbox{\parbox{\textwidth}{\citetext{bib:Rebeca1999}}}

\begin{verbatim}
@mastersthesis{bib:Rebeca1999,
  author = {Rebeca, Rosilene},
  title  = {Influência do ciclo estral no comportamento rotacional 
            em nado livre de camundongos suíços adultos},
  year-presented  = {1999},
  pages    = {79},
  pagename = {f.},
  type    = {Mestrado em Biologia},
  school  = {Instituto de Biologia Roberto Alcântara Gomes,
             Universidade do Estado do Rio de Janeiro},
  address = {Rio de Janeiro},
  year    = {1999}
}
\end{verbatim}

\noindent\fbox{\parbox{\textwidth}{\citetext{bib:Morgado1990}}}

\begin{verbatim}
@monography{bib:Morgado1990,
    author = {M. L. C. Morgado},
    title = {Reimplante dent\'ario},
    year-presented = {1990},
    pages = {51},
    pagename = {f.},
    type = {Gradua{\c c}\~ao em Odontologia},
    school = {Faculdade de Odontologia, 
              Universidade Camilo Castelo Branco},
    address = {São Paulo},
    year = {1990},
}
\end{verbatim}

\noindent\fbox{\parbox{\textwidth}{\citetext{bib:Silva2000}}}

\begin{verbatim}
@thesis{bib:Silva2000,
  author={N. C. Silva},
  title={Bibliotecas da UERJ},
  subtitle={proposta de um Centro Referencial baseado num estudo
            historiogr{\'a}fico},
  year-presented={2000},
  pages={161},
  pagename={f.},
  type={Trabalho de Conclusão de Curso (Especializa{\c c}{\~a}o
        em Organiza{\c c}{\~a}o do Conhecimento para 
        Recupera{\c c}{\~a}o da Informa{\c c}{\~a}o)},
  school={Escola de Biblioteconomia, 
          Universidade Federal do Estado do Rio de Janeiro},
  address={Rio de Janeiro},
  year={2000},
}
\end{verbatim}

%-----------------------------------------------------------------
\subsection{Partes de documentos}
%-----------------------------------------------------------------

%..............................................................
\subsubsection{Artigos em periódicos}
%..............................................................

\noindent\fbox{\parbox{\textwidth}{\citetext{bib:article}}}

\begin{verbatim}
@article{bib:article,
    author = {Sobrenome2, Nome2 and Sobrenome1, Nome1 and
              Sobrenome3, Nome3 and Sobrenome4, Nome4},
    title = {Título do artigo},
    journal = {Título da revista},
    address = {Local de publicação},
    volume = {Número do volume e/ou ano},
    number = {Número do fascículo},
    pages = {pág inicial -- pág final},
    month = {Mês},
    year = {Ano do fascículo}
}
\end{verbatim}

\noindent\textbf{Exemplo}\\

\noindent\fbox{\parbox{\textwidth}{\citetext{bib:Moura1983}}}

\begin{verbatim}
@article{bib:Moura1983,
  author = {Moura, Alexandrina Sobreira de},
  title = {Direito de habitação às classes de baixa renda},
  journal = {Ciência \& Trópico},
  address = {Recife},
  volume = {11},
  number = {1},
  pages = {71--78},
  month = {jan./jun.},
  year = {1983}
}
\end{verbatim}


%..............................................................
\subsubsection{Partes de documento (capítulo, volume, etc.)}
%..............................................................

\noindent\fbox{\parbox{\textwidth}{\citetext{bib:partedocumento}}}

\begin{verbatim}
@incollection{bib:partedocumento,
    author = {Autores da parte},
    title = {T{\'i}tulo da parte},
    subtitle = {Subt{\'i}tulo da parte},
    editor = {Corpo Editorial do documento},
    editortype = {Org. ou Ed.},
    booktitle = {T{\'i}tulo do documento},
    booksubtitle = {Subt{\'i}tulo do documento},
    edition = {Edição},
    address = {Local de publicação},
    publisher = {Editora},
    year = {Ano de publicação},
    volume = {Volume da parte referenciada},
    chapter = {Capítulo da parte referenciada},
    number = {Número da parte referenciada},
    pages = {pág. inicial -- pág. final},
}
\end{verbatim}

\noindent\fbox{\parbox{\textwidth}{\citetext{bib:partedocumento1}}}

\begin{verbatim}
@inbook{bib:partedocumento1,
    author = {Autores de parte do livro},
    title = {T{\'i}tulo de parte do livro},
    subtitle = {Subtítulo de parte do livro},
    booktitle = {T{\'i}tulo do livro},
    booksubtitle = {Subt{\'i}tulo do livro},
    furtherresp = {Responsabilidade},
    edition = {Edição},
    address = {Local de publicação},
    publisher = {Editora},
    year = {Ano da publicação},
    volume = {Volume da parte referenciada},
    chapter = {Capítulo da parte referenciada},
    number = {Número da parte referenciada},
    pages = {pág. inicial -- pág. final},
}
\end{verbatim}

\noindent\textbf{Exemplo}\\

\noindent\fbox{\parbox{\textwidth}{\citetext{bib:Ferreira2008}}}

\begin{verbatim}
@incollection{bib:Ferreira2008,
    author = {Ferreira, S. M. P.},
    title = {Repositórios \textsl{versus} revistas científicas},
    subtitle = {convergências e convivências},
    editor = {Ferreira, S. M. P. and Trajano, M. G.},
    booktitle = {Mais sobre revistas científicas},
    booksubtitle = {em foco a gestão},
    address = {São Paulo},
    publisher = {SENAC; CENGAGE Learning},
    year = {2008},
    pages = {111--137},
}
\end{verbatim}

\noindent\fbox{\parbox{\textwidth}{\citetext{bib:Ferreira2008}}}

\begin{verbatim}
@incollection{bib:Ferreira2008,
    author = {Arendt, H.},
    title = {A tradição e a época moderna},
    booktitle = {Entre o passado e o futuro},
    address = {São Paulo},
    publisher = {Perspectiva},
    year = {1992},
    pages = {43--68},
}
\end{verbatim}

%..............................................................
\subsubsection{Trabalhos apresentados em eventos (congressos, conferências, seminários, etc.)}
%..............................................................

\noindent\fbox{\parbox{\textwidth}{\citetext{bib:inproceedings}}}

\begin{verbatim}
@inproceedings{bib:inproceedings,
    author = {Sobrenome3, Nome3 and Sobrenome1, Nome1 and
              Sobrenome2, Nome2 and Sobrenome4, Nome4},
    title = {Título do trabalho},
    organization = {Nome do evento},
    conference-number = {Número do evento},
    conference-year = {Ano da realização},
    conference-location = {Local de realização},
    booktitle = {Título},
    address = {Local de publicação},
    publisher = {Editora},
    year = {Data de publicação},
    pages = {pág. inicial -- pág. final}
}
\end{verbatim}

\noindent\textbf{Exemplo}\\

\noindent\fbox{\parbox{\textwidth}{\citetext{bib:Silva2009}}}

\begin{verbatim}
@inproceedings{bib:Silva2009,
  author = {Silva, A. P. and Freire, J. R. B.},
  title = {Mem{\'o}ria oral e patrim{\^o}nio ind{\'i}gena no Brasil nas
           crônicas do s{\'e}culo XVI},
  organization = {SIMPÓSIO NACIONAL DE HISTÓRIA},
  conference-number = {36},
  conference-year = {2009},
  conference-location = {Fortaleza},
  booktitle = {História e Ética...},
  address = {Fortaleza},
  publisher = {UFC},
  year = {2009},
  url = {http://www.ifch.unicamp.br/ihb/ST36-Prog.htm},
  urlaccessdate = {23 nov. 2010},
}
\end{verbatim}

\noindent\fbox{\parbox{\textwidth}{\citetext{bib:Machado1998}}}

\begin{verbatim}
@inproceedings{bib:Machado1998,
  author = {Machado, Caio G. and Rodrigues, N{\'\i}vea M. R.},
  title = {Altera{\c c}{\~a}o de altura de forrageamento de esp{\'e}cies 
           de aves quando associadas a bandos mistos},
  organization = {Congresso Brasileiro de Ornitologia},
  conference-number = {7},
  conference-year = {1998},
  conference-location = {Rio de Janeiro},
  booktitle = {Resumos...},
  address = {Rio de Janeiro},
  publisher = {UERJ, NAPE},
  year = {1998},
  pages = {60-85}
}
\end{verbatim}

%-----------------------------------------------------------------
\subsection{Documentos em meio eletrônico}
%-----------------------------------------------------------------

%..............................................................
\subsubsection{Acesso online}
%..............................................................

A inserção de referência a documento acessado na modalidade \textsl{online} deve respeitar a natureza do documento e acrescentar, ao final, o endereço eletrônico para acesso e a data em que o documento foi consultado.\\

\noindent\fbox{\parbox{\textwidth}{\citetext{bib:online}}}

\begin{verbatim}
@book{bib:online,
    author = {Sobrenome1, Nome1 and Sobrenome2, Nome2},
    title = {T\'itulo do documento},
    edition = {Edição},
    address = {Local de publicação},
    publisher = {Editora},
    year = {Data de publicação},
    pages = {Número de páginas ou volumes},
    url = {http://endereco.eletronico},
    urlaccessdate = {Data de acesso}
}
\end{verbatim}

\noindent\textbf{Exemplo}\\

\noindent\fbox{\parbox{\textwidth}{\citetext{bib:Moura1996}}}

\begin{verbatim}
@book{bib:Moura1996,
    author = {Moura, Gevilacio Aguiar Coelho de},
    title = {Cita{\c c}{\~o}es e refer{\^e}ncias de documentos
             eletr{\^o}nicos},
    edition = {},
    address = {{[}S.l.},
    publisher = {s.n.},
    year = {19{---}{]}},
    pages = {86},
    url = {http://www.elogica.com.br/users/gmoura/reft},
    urlaccessdate = {9 dez. 1996}
}
\end{verbatim}

%..............................................................
\subsubsection{\textsl{e-book}}
%..............................................................

\noindent\fbox{\parbox{\textwidth}{\citetext{bib:ebook}}}

\begin{verbatim}
@book{bib:ebook,
    author = {Sobrenome1, Nome1 and Sobrenome2, Nome2 and
              Sobrenome3, Nome3},
    title = {Título do e-book},
    subtitle = {Subtítulo do e-book},
    edition = {Edição},
    address = {Local de publicação},
    publisher = {Editora},
    year = {Data da publicação},
    pages = {Número de paginas ou volumes (opcional)},
    pagename = {p.},
    note = {Mídia utilizada},
}
\end{verbatim}

\noindent\textbf{Exemplos}\\

\noindent\fbox{\parbox{\textwidth}{\citetext{bib:Lemos2010}}}

\begin{verbatim}
@book{bib:Lemos2010,
    author = {Lemos, A.},
    title = {Caderno de viagem},
    subtitle = {comunicação, lugares e tecnologias},
    address = {Porto Alegre},
    publisher = {Editora Plus},
    year = {2010},
    note = {E-book},
}
\end{verbatim}

\noindent\fbox{\parbox{\textwidth}{\citetext{bib:Pires2010}}}

\begin{verbatim}
@book{bib:Pires2010,
    author = {Pires, A.},
    title = {Direito tributário},
    address = {São Paulo},
    publisher = {Cia. das Letras},
    year = {2010},
    note = {Edição Kindle},
}
\end{verbatim}

%..............................................................
\subsubsection{\textsl{Homepage} institucional}
%..............................................................

\noindent\fbox{\parbox{\textwidth}{\citetext{bib:homepage}}}

\begin{verbatim}
@misc{bib:homepage,
    organization = {T{\'i}tulo da homepage},
    furtherresp = {Respons{\'a}vel (se houver)},
    howpublished  = {Descrição sucinta do conteúdo},
    url = {http://endereco.eletronico},
    urlaccessdate = {Data de acesso},
}
\end{verbatim}

\noindent\fbox{\parbox{\textwidth}{\citetext{bib:homepage1}}}

\begin{verbatim}
@misc{bib:homepage1,
    title = {T{\'i}tulo da homepage},
    furtherresp = {Respons{\'a}vel (se houver)},
    howpublished  = {Descrição sucinta do conteúdo},
    url = {http://endereco.eletronico},
    urlaccessdate = {Data de acesso},
}
\end{verbatim}
\noindent\textbf{Exemplos}\\

\noindent\fbox{\parbox{\textwidth}{\citetext{bib:Pinturabrasileira2005}}}

\begin{verbatim}
@misc{bib:Pinturabrasileira2005,
  title = {Arte e pintura brasileira: galeria virtual de arte},
  howpublished = {Apresenta reprodu{\c c}{\~o}es virtuais de pinturas 
                  brasileiras},
  url = {http://www.pinturabrasileira.com},
  urlaccessdate = {10 abr. 2005}
}
\end{verbatim}

\noindent\fbox{\parbox{\textwidth}{\citetext{bib:ufjf2006}}}

\begin{verbatim}
@misc{bib:ufjf2006,
  organization = {Universidade Federal de Juiz de Fora},
  furtherresp = {Desenvolvido por {C}idaeli {I}nformática {L}tda},
  howpublished = {Apresenta informa{\c c}{\~o}es gerais sobre a 
                  universidade},
  url = {http://www.ufjf.br},
  urlaccessdate = {15 maio 2006}
}
\end{verbatim}

\noindent\fbox{\parbox{\textwidth}{\citetext{bib:UERJ2011}}}

\begin{verbatim}
@misc{bib:UERJ2011,
    organization = {Universidade do Estado do Rio de Janeiro},
    furtherresp = {Desenvolvido pela Diretoria de Comunicação 
                   Social - COMUNS},
    howpublished  = {Apresenta informações gerais sobre a universidade},
    url = {http://www.uerj.br/},
    urlaccessdate = {10 abr. 2011},
}
\end{verbatim}

%..............................................................
\subsubsection{Programas (\textsl{software})}
%..............................................................

\noindent\fbox{\parbox{\textwidth}{\citetext{bib:software}}}

\begin{verbatim}
@book{bib:software,
    organization = {Autoria},
    title = {T\'itulo do programa},
    subtitle = {subtítulo},
    furtherresp = {Versão (se houver)},
    address = {Local de criação},
    publisher = {Editor (se houver)},
    year = {data de publicação},
    url = {http://endereco.eletronico},
    urlaccessdate = {Data de acesso},
}
\end{verbatim}

\noindent\textbf{Exemplos}\\

\noindent\fbox{\parbox{\textwidth}{\citetext{bib:Nourau2005}}}

\begin{verbatim}
@book{bib:Nourau2005,
    organization = {Universidade Estadual de Campinas},
    title = {Nou-Rau},
    subtitle = {software livre},
    furtherresp = {Versão beta 2},
    address = {Campinas},
    year = {2002},
    url = {http://www.rau-tu.unicamp.br/nou-rau/},
    urlaccessdate = {12 jan. 2005}
}
\end{verbatim}

%=====================================================================
\chapter*{Conclusão}
%=====================================================================

Texto da conclusão.

% ----------------------------------------------------------
% ELEMENTOS POS-TEXTUAIS
% ----------------------------------------------------------

\backmatter % marca o início do elementos pós-textuais

%=====================================================================
% Referencias via BibTeX
%=====================================================================

% citeoption indica o arquivo de opções, mas não o carrega
\citeoption{abnt-options4}
% a carga do arquivo de opções ocorre no bibliography
% insira sua bibliografia no arquivo bibliografia.bib
% exemplos.bib e modelos_bibtex.bib são arquivos com exemplos
% e modelos a partir das quais você poderá criar os seus
\bibliography{abnt-options4,bibliografia,exemplos,modelos_bibtex}

%=====================================================================
\postextualchapter*{Glossário}
%=====================================================================

\definicao{termo}{significado}
\definicao{termo}{significado}
\definicao{termo}{significado}

% ----------------------------------------------------------
% Apêndices (opcionais)
% ----------------------------------------------------------

\appendix % marca o início dos apêndices

%=====================================================================
\postextualchapter{Modelos em \textsf{BibTeX}}
%=====================================================================

%------------------------------------------------------------------
\section{Modelo $@$article} \label{sec:modelos}
%------------------------------------------------------------------

\noindent\fbox{\parbox{\textwidth}{\citetext{bib:article0}}}

\begin{verbatim}
@article{bib:article0,
    author = {Autores do artigo},
    title = {T{\'i}tulo do artigo},
    subtitle = {Subtítulo do artigo},
    journal = {Título do periódico},
    section = {Seção},
    publisher = {Editora},
    address = {Local},
    volume = {Volume},
    number = {Número},
    pages = {pág. inicial -- pág. final},
    month = {Mês abreviado},
    year = {Ano da publicação},
    note = {Anotações},
    url = {http://endereco.eletronico},
}
\end{verbatim}

\noindent\fbox{\parbox{\textwidth}{\citetext{bib:article1}}}\\

\begin{verbatim}
@article{bib:article1,
    organization = {Organiza{\c c}{\~a}o respons{\'a}vel pelo artigo},
    org-short = {Sigla},
    title = {T{\'i}tulo do artigo},
    subtitle = {Subtítulo do artigo},
    journal = {Título do periódico},
    section = {Seção},
    publisher = {Editora},
    address = {Local de publicação},
    volume = {Número do volume e/ou ano},
    number = {número do fascículo},
    pages = {pág. inicial -- pág. final},
    month = {Mês abreviado},
    year = {Ano do fascículo},
    note = {Anotações},
    url = {http://endereco.eletronico},
}
\end{verbatim}

%------------------------------------------------------------------
\section{Modelo $@$book}
%------------------------------------------------------------------

\noindent\fbox{\parbox{\textwidth}{\citetext{bib:book0}}}

\begin{verbatim}
@book{bib:book0,
    author = {Autores do livro},
    type = {Tipo},
    title = {T{\'i}tulo do livro},
    subtitle = {Subtítulo do livro},
    furtherresp = {Nota},
    edition = {Edição},
    address = {Local de publicação},
    publisher = {Editora},
    year = {Ano da publicação},
    pages = {Número de páginas},
    howpublished = {Comentário},
    series = {Série},
    number = {Número da publicação},
    note = {Anotação},
    url = {http://endereco.eletronico},
}
\end{verbatim}

\noindent\fbox{\parbox{\textwidth}{\citetext{bib:book1}}}

\begin{verbatim}
@book{bib:book1,
    editor = {Corpo Editorial do livro},
    type = {Tipo},
    title = {T{\'i}tulo do livro},
    subtitle = {Subtítulo do livro},
    furtherresp = {Nota},
    edition = {Edição},
    address = {Local de publicação},
    publisher = {Editora},
    year = {Ano da publicação},
    pages = {Número de páginas},
    howpublished = {Comentário},
    series = {Série},
    number = {Número da publicação},
    note = {Anotação},
    url = {http://endereco.eletronico}
}
\end{verbatim}

\noindent\fbox{\parbox{\textwidth}{\citetext{bib:book2}}}

\begin{verbatim}
@book{bib:book2,
    organization = {Organiza{\c c}{\~a}o respons{\'a}vel pelo livro},
    org-short = {Sigla},
    type = {Tipo},
    title = {T{\'i}tulo do livro},
    subtitle = {Subtítulo do livro},
    furtherresp = {Nota},
    edition = {Edição},
    address = {Local de publicação},
    publisher = {Editora},
    year = {Ano da publicação},
    pages = {Número de páginas},
    howpublished = {Comentário},
    series = {Série},
    number = {Número da publicação},
    note = {Anotação},
    url = {http://endereco.eletronico},
}
\end{verbatim}

\noindent\fbox{\parbox{\textwidth}{\citetext{bib:book3}}}\\

\begin{verbatim}
@book{bib:book3,
    type = {Tipo},
    title = {T{\'i}tulo do livro},
    subtitle = {Subtítulo do livro},
    furtherresp = {Nota},
    edition = {Edição},
    address = {Local de publicação},
    publisher = {Editora},
    year = {Ano da publicação},
    pages = {Número de páginas},
    howpublished = {Comentário},
    series = {Série},
    number = {Número da publicação},
    note = {Anotação},
    url = {http://endereco.eletronico},
}
\end{verbatim}

%------------------------------------------------------------------
\section{Modelo $@$manual}
%------------------------------------------------------------------

\noindent\fbox{\parbox{\textwidth}{\citetext{bib:manual0}}}

\begin{verbatim}
@manual{bib:manual0,
    author = {Autores do manual},
    title = {T{\'i}tulo do manual},
    subtitle = {Sub{\'i}tulo do manual},
    furtherresp = {Nota},
    edition = {Edição},
    address = {Local de publicação},
    year = {Ano da publicação},
    pages = {Número de páginas},
    series = {Série},
    number = {Número da publicação},
    note = {Anotações},
    url = {http://endereco.eletronico},
}
\end{verbatim}

\noindent\fbox{\parbox{\textwidth}{\citetext{bib:manual1}}}

\begin{verbatim}
@manual{bib:manual1,
    editor = {Corpo Editorial do manual},
    title = {T{\'i}tulo do manual},
    subtitle = {Sub{\'i}tulo do manual},
    furtherresp = {Nota},
    edition = {Edição},
    address = {Local de publicação},
    year = {Ano da publicação},
    pages = {Número de páginas},
    series = {Série},
    number = {Número da publicação},
    note = {Anotações},
    url = {http://endereco.eletronico},
}
\end{verbatim}

\noindent\fbox{\parbox{\textwidth}{\citetext{bib:manual2}}}

\begin{verbatim}
@manual{bib:manual2,
    organization = {Organiza{\c c}{\~a}o respons{\'a}vel pelo manual},
    org-short = {Sigla},
    title = {T{\'i}tulo do manual},
    subtitle = {Sub{\'i}tulo do manual},
    furtherresp = {Nota},
    edition = {Edição},
    address = {Local de publicação},
    year = {Ano da publicação},
    pages = {Número de páginas},
    series = {Série},
    number = {Número da publicação},
    note = {Anotações},
    url = {http://endereco.eletronico},
}
\end{verbatim}

\noindent\fbox{\parbox{\textwidth}{\citetext{bib:manual3}}}\\

\begin{verbatim}
@manual{bib:manual3,
    title = {T{\'i}tulo do manual},
    subtitle = {Sub{\'i}tulo do manual},
    furtherresp = {Nota},
    edition = {Edição},
    address = {Local de publicação},
    year = {Ano da publicação},
    pages = {Número de páginas},
    series = {Série},
    number = {Número da publicação},
    note = {Anotações},
    url = {http://endereco.eletronico},
}
\end{verbatim}

%------------------------------------------------------------------
\section{Modelo $@$techreport}
%------------------------------------------------------------------

\noindent\fbox{\parbox{\textwidth}{\citetext{bib:techreport0}}}

\begin{verbatim}
@techreport{bib:techreport0,
    author = {Autores do Relat{\'o}rio T{\'e}cnico},
    title = {T{\'i}tulo do relatório técnico},
    subtitle = {Subtítulo do relatório técnico},
    furtherresp = {Nota},
    edition = {Edição},
    address = {Local de publicação},
    year = {Ano da publicação},
    pages = {Número de páginas},
    series = {Série},
    number = {Número da publicação},
    note = {Anotação},
    url = {http://endereco.eletronico},

    type = {Tipo},
    publisher = {Editora},
    howpublished = {Comentário},
    volume = {Volume da publicação},
    month = {Mês abreviado},
}
\end{verbatim}

\noindent\fbox{\parbox{\textwidth}{\citetext{bib:techreport1}}}

\begin{verbatim}
@techreport{bib:techreport1,
    editor = {Corpo Editorial do Relat{\'o}rio T{\'e}cnico},
    title = {T{\'i}tulo do relatório técnico},
    subtitle = {Subtítulo do relatório técnico},
    furtherresp = {Nota},
    edition = {Edição},
    address = {Local de publicação},
    year = {Ano da publicação},
    pages = {Número de páginas},
    series = {Série},
    number = {Número da publicação},
    note = {Anotação},
    url = {http://endereco.eletronico},
}
\end{verbatim}

\noindent\fbox{\parbox{\textwidth}{\citetext{bib:techreport2}}}

\begin{verbatim}
@techreport{bib:techreport2,
    organization = {Organiza{\c c}{\~a}o respons{\'a}vel 
    				pelo relat{\'o}rio t{\'e}cnico},
    org-short = {Sigla},
    title = {T{\'i}tulo do relatório técnico},
    subtitle = {Subtítulo do relatório técnico},
    furtherresp = {Nota},
    edition = {Edição},
    address = {Local de publicação},
    year = {Ano da publicação},
    pages = {Número de páginas},
    series = {Série},
    number = {Número da publicação},
    note = {Anotação},
    url = {http://endereco.eletronico},
}
\end{verbatim}

\noindent\fbox{\parbox{\textwidth}{\citetext{bib:techreport3}}}\\

\begin{verbatim}
@techreport{bib:techreport3,
    title = {T{\'i}tulo do relatório técnico},
    subtitle = {Subtítulo do relatório técnico},
    furtherresp = {Nota},
    edition = {Edição},
    address = {Local de publicação},
    year = {Ano da publicação},
    pages = {Número de páginas},
    series = {Série},
    number = {Número da publicação},
    note = {Anotação},
    url = {http://endereco.eletronico},
}
\end{verbatim}

%------------------------------------------------------------------
\section{Modelo $@$inbook}
%------------------------------------------------------------------

\noindent\fbox{\parbox{\textwidth}{\citetext{bib:inbook0}}}

\begin{verbatim}
@inbook{bib:inbook0,
    author = {Autores de parte do livro},
    title = {T{\'i}tulo de parte do livro},
    subtitle = {Subtítulo de parte do livro},
    booktitle = {T{\'i}tulo do livro},
    booksubtitle = {Subt{\'i}tulo do livro},
    furtherresp = {Nota},
    edition = {Edição},
    address = {Local de publicação},
    publisher = {Editora},
    year = {Ano da publicação},
    series = {Série},
    number = {Número da publicação},
    chapter = {Capítulo},
    pages = {pág. inicial -- pág. final},
    note = {Anotação},
    url = {http://endereco.eletronico},
}
\end{verbatim}

\noindent\fbox{\parbox{\textwidth}{\citetext{bib:inbook1}}}

\begin{verbatim}
@inbook{bib:inbook1,
    organization = {Organiza{\c c}{\~a}o respons{\'a}vel 
    				pela parte do livro},
    org-short = {Sigla},
    title = {T{\'i}tulo de parte do livro},
    subtitle = {Subt{\'i}tulo de parte do livro},
    editor = {Corpo Editorial do livro},
    booktitle = {T{\'i}tulo do livro},
    booksubtitle = {Subt{\'i}tulo do livro},
    furtherresp = {Nota},
    edition = {Edição},
    address = {Local de publicação},
    publisher = {Editora},
    year = {Ano da publicação},
    series = {Série},
    number = {Número da publicação},
    chapter = {Capítulo},
    pages = {pág. inicial -- pág. final},
    note = {Anotação},
    url = {http://endereco.eletronico},
}
\end{verbatim}

\noindent\fbox{\parbox{\textwidth}{\citetext{bib:inbook2}}}

\begin{verbatim}
@inbook{bib:inbook2,
    title = {T{\'i}tulo de parte do livro},
    subtitle = {Subt{\'i}tulo de parte do livro},
    editor = {Corpo Editorial do livro},
    booktitle = {T{\'i}tulo do livro},
    booksubtitle = {Subt{\'i}tulo do livro},
    furtherresp = {Nota},
    edition = {Edição},
    address = {Local de publicação},
    publisher = {Editora},
    year = {Ano de publicação},
    series = {Série},
    number = {Número da publicação},
    chapter = {Capítulo},
    pages = {pág. inicial -- pág. final},
    note = {Anotação},
    url = {http://endereco.eletronico},
}
\end{verbatim}

\noindent\fbox{\parbox{\textwidth}{\citetext{bib:inbook3}}}\\

\begin{verbatim}
@inbook{bib:inbook3,
    title = {T{\'i}tulo de parte do livro},
    subtitle = {Subt{\'i}tulo de parte do livro},
    booktitle = {T{\'i}tulo do livro},
    booksubtitle = {Subt{\'i}tulo do livro},
    furtherresp = {Nota},
    edition = {Edição},
    address = {Local de publicação},
    publisher = {Editora},
    year = {Ano da publicação},
    series = {Série},
    number = {Número de publicação},
    chapter = {Capítulo},
    pages = {pág. inicial -- pág. final},
    note = {Anotação},
    url = {http://endereco.eletronico},
}
\end{verbatim}

%------------------------------------------------------------------
\section{Modelo $@$incollection}
%------------------------------------------------------------------

\noindent\fbox{\parbox{\textwidth}{\citetext{bib:incollection0}}}

\begin{verbatim}
@incollection{bib:incollection0,
    author = {Autores do documento},
    title = {T{\'i}tulo do documento},
    subtitle = {Subt{\'i}tulo do documento},
    editor = {Corpo Editorial da cole{\c c}{\~a}o},
    booktitle = {T{\'i}tulo da cole{\c c}{\~a}o},
    booksubtitle = {Subt{\'i}tulo da cole{\c c}{\~a}o},
    edition = {Edição},
    address = {Local de publicação},
    publisher = {Editora},
    year = {Ano de publicação},
    series = {Série},
    number = {Número da publicação},
    chapter = {Capítulo},
    pages = {pág. inicial -- pág. final},
    note = {Anotações},
    url = {http://endereco.eletronico}
}
\end{verbatim}

\noindent\fbox{\parbox{\textwidth}{\citetext{bib:incollection1}}}

\begin{verbatim}
@incollection{bib:incollection1,
    organization = {Organiza{\c c}{\~a}o respons{\'a}vel pelo documento},
    org-short = {Sigla},
    title = {T{\'i}tulo do documento},
    subtitle = {Subt{\'i}tulo do documento},
    editor = {Corpo Editorial da cole{\c c}{\~a}o},
    booktitle = {T{\'i}tulo da cole{\c c}{\~a}o},
    edition = {Edição},
    address = {Local de publicação},
    publisher = {Editora},
    year = {Ano de publicação},
    series = {Série},
    number = {Número da publicação},
    chapter = {Capítulo},
    pages = {pág. inicial -- pág. final},
    note = {Anotações},
    url = {http://endereco.eletronico}
}
\end{verbatim}

\noindent\fbox{\parbox{\textwidth}{\citetext{bib:incollection2}}}

\begin{verbatim}
@incollection{bib:incollection2,
    editor = {Corpo Editorial da cole{\c c}{\~a}o},
    title = {T{\'i}tulo do documento},
    subtitle = {Subt{\'i}tulo do documento},
    booktitle = {T{\'i}tulo da cole{\c c}{\~a}o},
    booksubtitle = {Subt{\'i}tulo da cole{\c c}{\~a}o},
    edition = {Edição},
    address = {Local da publicação},
    publisher = {Editora},
    year = {Ano da publicação},
    series = {Série},
    number = {Número da publicação},
    chapter = {Capítulo},
    pages = {pág. inicial -- pág. final},
    note = {Anotações},
    url = {http://endereco.eletronico},
}
\end{verbatim}

\noindent\fbox{\parbox{\textwidth}{\citetext{bib:incollection3}}}\\

\begin{verbatim}
@incollection{bib:incollection3,
    title = {T{\'i}tulo do documento},
    subtitle = {Subt{\'i}tulo do documento},
    booktitle = {T{\'i}tulo da cole{\c c}{\~a}o},
    booksubtitle = {Subt{\'i}tulo da cole{\c c}{\~a}o},
    edition = {Edição},
    address = {Local da publicação},
    publisher = {Editora},
    year = {Ano da publicação},
    series = {Série},
    number = {Número da publicação},
    chapter = {Capítulo},
    pages = {pág. inicial -- pág. final},
    note = {Anotações},
    url = {http://endereco.eletronico},
}
\end{verbatim}

%------------------------------------------------------------------
\section{Modelo $@$inproceedings}
%------------------------------------------------------------------

\noindent\fbox{\parbox{\textwidth}{\citetext{bib:inproceedings0}}}

\begin{verbatim}
@inproceedings{bib:inproceedings0,
    author = {Autores do trabalho},
    title = {T{\'i}tulo do trabalho},
    subtitle = {Subt{\'i}tulo do trabalho},
    editor = {Corpo Editorial dos proceedings},
    booktitle = {T{\'i}tulo dos proceedings},
    booksubtitle = {Subt{\'i}tulo dos proceedings},
    address = {Local de publicação},
    publisher = {Editora},
    year = {Ano de publicação},
    series = {Série},
    number = {Número da publicação},
    pages = {pág. inicial -- pág. final},
    note = {Anotações},
    url = {http://endereco.eletronico},
}
\end{verbatim}

\noindent\fbox{\parbox{\textwidth}{\citetext{bib:inproceedings1}}}

\begin{verbatim}
@inproceedings{bib:inproceedings1,
    organization = {Organiza{\c c}{\~a}o respons{\'a}vel pelo trabalho},
    org-short = {Sigla},
    title = {T{\'i}tulo do trabalho},
    subtitle = {Subt{\'i}tulo do trabalho},
    editor = {Corpo Editorial dos proceedings},
    booktitle = {T{\'i}tulo dos proceedings},
    booksubtitle = {Subt{\'i}tulo dos proceedings},
    address = {Local de publicação},
    publisher = {Editora},
    year = {Ano de publicação},
    series = {Série},
    number = {Número da publicação},
    pages = {pág. inicial -- pág. final},
    note = {Anotações},
    url = {http://endereco.eletronico},
}
\end{verbatim}

\noindent\fbox{\parbox{\textwidth}{\citetext{bib:inproceedings2}}}

\begin{verbatim}
@inproceedings{bib:inproceedings2,
    title = {T{\'i}tulo do trabalho},
    subtitle = {Subt{\'i}tulo do trabalho},
    editor = {Corpo Editorial dos proceedings},
    booktitle = {T{\'i}tulo dos proceedings},
    booksubtitle = {Subt{\'i}tulo dos proceedings},
    address = {Local de publicação},
    publisher = {Editora},
    year = {Ano de publicação},
    series = {Série},
    number = {Número da publicação},
    pages = {pág. inicial -- pág. final},
    note = {Anotações},
    url = {http://endereco.eletronico},
}
\end{verbatim}

\noindent\fbox{\parbox{\textwidth}{\citetext{bib:inproceedings3}}}\\

\begin{verbatim}
@inproceedings{bib:inproceedings3,
    title = {T{\'i}tulo do trabalho},
    subtitle = {Subt{\'i}tulo do trabalho},
    booktitle = {T{\'i}tulo dos proceedings},
    booksubtitle = {Subt{\'i}tulo dos proceedings},
    address = {Local de publicação},
    publisher = {Editora},
    year = {Ano de publicação},
    series = {Série},
    number = {Número da publicação},
    pages = {pág. inicial -- pág. final},
    note = {Anotações},
    url = {http://endereco.eletronico},
}
\end{verbatim}

%------------------------------------------------------------------
\section{Modelo $@$proceedings}
%------------------------------------------------------------------

\noindent\fbox{\parbox{\textwidth}{\citetext{bib:proceedings0}}}

\begin{verbatim}
@proceedings{bib:proceedings0,
    editor = {Corpo Editorial dos proceedings},
    title = {T{\'i}tulo dos proceedings},
    subtitle = {Subt{\'i}tulo dos proceedings},
    volume = {Volume},
    number = {Número},
    series = {Série},
    organization = {Organiza{\c c}{\~a}o respons{\'a}vel pelos proceedings},
    org-short = {Sigla},
    address = {Local de publicação},
    publisher = {Editora},
    year = {Ano de publicação},
    pages = {Número de páginas},
    note = {Anotações},
    url = {http://endereco.eletronico},
}
\end{verbatim}

\noindent\fbox{\parbox{\textwidth}{\citetext{bib:proceedings1}}}

\begin{verbatim}
@proceedings{bib:proceedings1,
    organization = {Organiza{\c c}{\~a}o respons{\'a}vel pelos proceedings},
    org-short = {Sigla},
    title = {T{\'i}tulo dos proceedings},
    subtitle = {Subt{\'i}tulo dos proceedings},
    volume = {Volume},
    number = {Número},
    series = {Série},
    address = {Local de publicação},
    publisher = {Editora},
    year = {Ano de publicação},
    pages = {Número de páginas},
    note = {Anotações},
    url = {http://endereco.eletronico},
}
\end{verbatim}

\noindent\fbox{\parbox{\textwidth}{\citetext{bib:proceedings2}}}\\

\begin{verbatim}
@proceedings{bib:proceedings2,
    title = {T{\'i}tulo dos proceedings},
    subtitle = {Subt{\'i}tulo dos proceedings},
    volume = {Volume},
    number = {Número},
    series = {Série},
    address = {Local de publicação},
    publisher = {Editora},
    year = {Ano de publicação},
    pages = {Número de páginas},
    note = {Anotações},
    url = {http://endereco.eletronico},
}
\end{verbatim}

%------------------------------------------------------------------
\section{Modelo $@$misc}
%------------------------------------------------------------------

\noindent\fbox{\parbox{\textwidth}{\citetext{bib:misc0}}}

\begin{verbatim}
@misc{bib:misc0,
    author = {Autor},
    type = {Tipo},
    title = {T{\'i}tulo},
    subtitle = {Subt{\'i}tulo},
    furtherresp = {Complemento},
    address = {Local},
    publisher = {Editora},
    month = {Mês abreviado},
    year = {Ano da publicação},
    pages = {Número de páginas},
    howpublished = {Comentário},
    series = {Série},
    number = {Número da publicação},
    note = {Anotação},
    url = {http://endereco.eletronico},
}
\end{verbatim}

\noindent\fbox{\parbox{\textwidth}{\citetext{bib:misc1}}}

\begin{verbatim}
@misc{bib:misc1,
    editor = {Editor},
    type = {Tipo},
    title = {T{\'i}tulo},
    subtitle = {Subt{\'i}tulo},
    furtherresp = {Complemento},
    address = {Local},
    publisher = {Editora},
    month = {Mês abreviado},
    year = {Ano da publicação},
    pages = {Número de páginas},
    howpublished = {Comentário},
    series = {Série},
    number = {Número da publicação},
    note = {Anotação},
    url = {http://endereco.eletronico},
}
\end{verbatim}

\noindent\fbox{\parbox{\textwidth}{\citetext{bib:misc2}}}

\begin{verbatim}
@misc{bib:misc2,
    organization = {Organiza{\c c}{\~a}o},
    org-short = {Sigla},
    type = {Tipo},
    title = {T{\'i}tulo},
    subtitle = {Subt{\'i}tulo},
    furtherresp = {Complemento},
    address = {Local},
    publisher = {Editora},
    month = {Mês abreviado},
    year = {Ano da publicação},
    pages = {Número de páginas},
    howpublished = {Comentário},
    series = {Série},
    number = {Número da publicação},
    note = {Anotação},
    url = {http://endereco.eletronico},
}
\end{verbatim}

\noindent\fbox{\parbox{\textwidth}{\citetext{bib:misc3}}}\\

\begin{verbatim}
@misc{bib:misc3,
    type = {Tipo},
    title = {T{\'i}tulo},
    subtitle = {Subt{\'i}tulo},
    furtherresp = {Complemento},
    address = {Local},
    publisher = {Editora},
    month = {Mês abreviado},
    year = {Ano da publicação},
    pages = {Número de páginas},
    howpublished = {Comentário},
    series = {Série},
    number = {Número da publicação},
    note = {Anotação},
    url = {http://endereco.eletronico},
}
\end{verbatim}

%------------------------------------------------------------------
\section{Modelo $@$unpublished}
%------------------------------------------------------------------

\noindent\fbox{\parbox{\textwidth}{\citetext{bib:unpublished0}}}

\begin{verbatim}
@unpublished{bib:unpublished0,
    author = {Autor de trabalho não publicado},
    title = {T{\'i}tulo},
    subtitle = {Subt{\'i}tulo},
    note = {Anotação},
    year = {Ano da publicação},
    url = {http://endereco.eletronico},
}
\end{verbatim}

\noindent\fbox{\parbox{\textwidth}{\citetext{bib:unpublished1}}}

\begin{verbatim}
@unpublished{bib:unpublished1,
    editor = {Editor de trabalho não publicado},
    title = {T{\'i}tulo},
    subtitle = {Subt{\'i}tulo},
    note = {Anotação},
    year = {Ano da publicação},
    url = {http://endereco.eletronico},
}
\end{verbatim}

\noindent\fbox{\parbox{\textwidth}{\citetext{bib:unpublished2}}}

\begin{verbatim}
@unpublished{bib:unpublished2,
    organization = {Organiza{\c c}{\~a}o respons{\'a}vel pelo trabalho
                    n{\~a}o publicado},
    org-short = {Sigla},
    title = {T{\'i}tulo},
    subtitle = {Subt{\'i}tulo},
    note = {Anotação},
    year = {Ano da publicação},
    url = {http://endereco.eletronico},
}
\end{verbatim}

\noindent\fbox{\parbox{\textwidth}{\citetext{bib:unpublished3}}}\\

\begin{verbatim}
@unpublished{bib:unpublished3,
    title = {T{\'i}tulo},
    subtitle = {Subt{\'i}tulo},
    note = {Anotação},
    year = {Ano da publicação},
    url = {http://endereco.eletronico},
}
\end{verbatim}

%=====================================================================
\postextualchapter{Segundo apêndice}
%=====================================================================

\section{Primeira seção}

Texto da primeira seção.

\subsection{Primeira subseção}

Texto da primeira subseção.

\subsubsection{Primeira subsubseção}

Texto da primeira subsubseção.

% ----------------------------------------------------------
% Anexos (opcionais)
% ----------------------------------------------------------

\annex % marca o início dos anexos

%=====================================================================
\postextualchapter{Primeiro anexo}
%=====================================================================

\section{Primeira seção}

Texto da primeira seção.

\subsection{Primeira subseção}

Texto da primeira subseção.

\subsubsection{Primeira subsubseção}

Texto da primeira subsubseção.

%=====================================================================
\postextualchapter{Segundo anexo}
%=====================================================================

\section{Primeira seção}

Texto da primeira seção.

\subsection{Primeira subseção}

Texto da primeira subseção.

\subsubsection{Primeira subsubseção}

Texto da primeira subsubseção.

%---------------------------------------------------------------------
% INDICE REMISSIVO (relativo ao makeindex)
%---------------------------------------------------------------------

\printindex

% ********************************************************************
% ********************************************************************
\end{document}
% ********************************************************************
% ********************************************************************