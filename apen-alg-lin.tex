\section{Alguns Conceitos e Definicoes em Algebra Linear}
Esta secao do apendice se destina a apresentar algumas ferramentas da algebra linear necessarias ao entendimento de algumas secoes da dissertacao, funcionando como complementacao dessas secoes.

\subsection{A Matriz Anti-Simétrica $[{\bf v}]_\times$}
Dado um vetor ${\bf v}=(v_1,v_2,v_3)^\top$ é possível construir uma matriz anti-simétrica com as componentes de ${\bf v}$ na forma

\begin{equation*}
[{\bf v}]_\times=
\begin{bmatrix}
0&-v_3&v_2\\
v_3&0&-v_1\\
-v_2&v_1&0
\end{bmatrix},
\end{equation*}
onde a notação $[{\bf v}]_\times$ indica o produto vetorial entre um vetor qualquer ${\bf u}$ e ${\bf v}$. 

\begin{equation*}
{\bf v} \times {\bf u} = [{\bf v}]_\times {\bf u} = ({\bf v}^\top [{\bf u}]_\times)^\top
\end{equation*}

Sabemos que o produto vetorial entre um vetor e ele mesmo é sempre o vetor nulo, portanto o vetor ${\bf v}$ é o vetor nulo à direita e à esquerda de $[{\bf v}]_\times$. Desta forma, esta matriz anti-simétrica será sempre definida por seu vetor nulo.  

\subsection{A Pseudo-Inversa de $P$}
