\begin{center}
{\bf RESUMO}
\end{center}

\vspace{1 cm}

\begin{center}
\begin{minipage}{1\textwidth}
\setstretch{1}
\noindent PINHO, D. C. {\it Geometria trifocal em reconstrução 3D.} Dissertação (Mestrado em Modelagem Computacional) - Instituto Politécnico do Rio de Janeiro, Universidade do Estado do Rio de Janeiro, Nova Friburgo, 2016.
\end{minipage}
\end{center}

\vspace{1 cm}

\begin{center}
\begin{minipage}{1\textwidth}
\setstretch{1}
\qquad Neste trabalho nós investigamos alguns benefícios na utilização da geometria trifocal aplicada em sistemas de reconstrução 3D multivisão e estimação de modelos de múltiplas câmeras, em oposição com a utilização da geometria epipolar com par de câmeras. Nós apresentamos a interpretação e o detalhamento matemático dos pontos mais importantes de dois artigos recentes e essenciais, objetivando resolver problemas trifocais de estrutura de curvas no futuro. Equanto um deles mostra a aplicação do Quatérnion de Hamilton e das Bases de Gr\"obner para a computação da pose de uma câmera dadas estruturas correspondentes 3D-2D, o outro é a abordagem mais eficiente computacionalmente (até a presente data) na reconstrução das câmeras num sistema trifocal. Os primeiros capítulos e os apêndices apresentam as teorias básicas sobre Geometria Projetiva, Álgebra Linear e Geometria Algébrica necessárias ao avanço das pesquisas no campo da reconstrução 3D multivisão e sistemas de determinação de estruturas a partir do movimento.
\end{minipage}
\end{center}
 
\vspace{1 cm}

\begin{flushleft}
Palavras-chave: Visão computacional. Reconstrução 3D. Geometria trifocal. Bases de Gr\"obner. 
\end{flushleft}

\newpage