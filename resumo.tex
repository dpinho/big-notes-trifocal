\begin{center}
{\bf RESUMO}
\end{center}

\vspace{1.5 cm}

\begin{center}
\begin{minipage}{1\textwidth}
\setstretch{1}
\noindent PINHO, David da Costa de. {\it Geometria trifocal em reconstrução 3D.} 2016. 128 f. Dissertação (Mestrado em Modelagem Computacional) - Instituto Politécnico, Universidade do Estado do Rio de Janeiro, Nova Friburgo, 2016.
\end{minipage}
\end{center}

\begin{center}
\begin{minipage}{1\textwidth}
\setstretch{1}
\qquad Neste trabalho nós investigamos alguns benefícios na utilização da geometria trifocal aplicada em sistemas de reconstrução 3D multifocal e estimação de modelos de múltiplas câmeras, em oposição à utilização da geometria epipolar com pares de câmeras. Apresentamos a interpretação e o detalhamento matemático dos pontos mais importantes de dois artigos recentes e essenciais, objetivando resolver problemas trifocais de estrutura de curvas no futuro. Equanto um deles mostra a aplicação do quatérnion de Hamilton e das bases de Gr\"obner para a cálculo da pose de uma câmera dadas estruturas correspondentes 3D--2D, o outro é a abordagem mais eficiente computacionalmente (até a presente data) na reconstrução das câmeras num sistema trifocal. Os primeiros capítulos e os apêndices apresentam as teorias básicas sobre geometria projetiva, álgebra linear e geometria algébrica necessárias ao avanço das pesquisas no campo da reconstrução 3D trifocal e multifocal e sistemas de determinação de estruturas a partir do movimento.
\end{minipage}
\end{center}
 
\begin{flushleft}
Palavras-chave: Visão computacional. Reconstrução 3D. Geometria trifocal. Bases de 
\end{flushleft}\vspace{-.5 cm}
\qquad \qquad \qquad \,\,\,\,\, Gr\"obner.
\newpage