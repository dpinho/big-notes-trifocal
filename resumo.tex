\begin{center}
{\bf RESUMO}
\end{center}

\vspace{1 cm}

\begin{center}
\begin{minipage}{1\textwidth}
\setstretch{1}
\noindent PINHO, D. C. {\it Geometria trifocal em reconstrução 3D.} Dissertação (Mestrado em Modelagem Computacional) - Instituto Politécnico do Rio de Janeiro, Universidade do Estado do Rio de Janeiro, Nova Friburgo, 2016.
\end{minipage}
\end{center}

\vspace{1 cm}

\begin{center}
\begin{minipage}{1\textwidth}
\setstretch{1}
\qquad Neste trabalho são apresentados alguns benefícios na utilização da geometria trifocal aplicada em transferência de pontos e reconstrução de câmeras, em comparação com a utilização da geometria epipolar num sistema com três visões. Há o detalhamento matemático dos pontos mais importantes de dois artigos. Equanto um deles mostra a aplicação do Quatérnion de Hamilton e das Bases de Gr\"obner para a obtenção de uma câmera, o outro é a abordagem mais eficiente (pelo menos até 2006) na reconstrução das câmeras num sistema trifocal. Nos primeiros capítulos e nos apêndices são disponibilizadas as teorias básicas sobre Geometria Projetiva, Álgebra Linear e Geometria Algébrica necessárias ao entendimento da dissertação.
\end{minipage}
\end{center}
 
\vspace{1 cm}

\begin{flushleft}
Palavras-chave: Visão computacional. Reconstrução 3D. Geometria trifocal. Bases de Gr\"obner. 
\end{flushleft}

\newpage