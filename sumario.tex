\begin{flushleft}
\begin{tabular}{llr}
&\hspace{6 cm}{\bf SUMÁRIO}&\\\\\\
  & {\bf INTRODUÇÃO} & 8\\ 
1 & {\bf GEOMETRIA DE UMA E DUAS C\^AMERAS}..................................&14 \\
1.1 & {\bf O Espaço Projetivo em Duas Dimensões}..............................................& 14 \\
1.1.1 & \underline{A Reta}...........................................................................................................& 14 \\
1.1.2 & \underline{O Ponto}......................................................................................................... & 14 \\
1.1.3 & \underline{A C\^onica}....................................................................................................... & 18 \\
1.1.4 & \underline{C\^onica Dual ou Reta}..................................................................................... & 20 \\
1.1.5 & \underline{C\^onicas Degeneradas}..................................................................................... & 20 \\
1.1.6 & \underline{A Relação Polo-Polar}..................................................................................... & 21 \\
1.1.7 & \underline{Transformações Projetivas em $\mathbb{P}^2$}.................................................................. & 23 \\
1.1.8 & \underline{A Geometria Projetiva de Uma Dimens\~ao}.................................................... & 24 \\
1.1.9 & \underline{Subgrupo de Transformações Projetivas}........................................................ & 27 \\
1.2 & {\bf O Espaço Projetivo em Três Dimensões}.............................................. & 30\\
1.2.1 & \underline{A Transformação de Pontos e Planos em ${\mathbb{P}^3}$}................................................. & 33\\
1.2.2 & \underline{Direções e o Plano no Infinito}....................................................................... & 33\\
1.2.3 & \underline{A Cônica Absoluta}........................................................................................ & 34\\
1.2.4 & \underline{A Quádrica Dual Absoluta}............................................................................ & 36\\
1.3 & {\bf A Câmera $P$}............................................................................................... & 37\\
1.4 & {\bf A Ação Projetiva de uma Câmera $P$}.................................................... & 44\\
1.4.1 & \underline{A Ação Projetiva de $P$ em Retas}.................................................................. & 46\\
1.4.2 & \underline{A Ação Projetiva de $P$ em Quádricas}........................................................... & 47\\
1.5 & {\bf A Geometria Bifocal}................................................................................. & 49\\
1.5.1 & \underline{A Homografia Induzida por um Plano}.......................................................... & 50\\
1.5.2 & \underline{A Matriz Fundamental $F$}.............................................................................. & 52\\
1.5.3 & \underline{Propriedades da Matriz Fundamental}........................................................... & 55\\
1.5.4 & \underline{Configuração Canônica das Câmeras}............................................................ & 56\\
1.5.5 & \underline{Ambiguidade Projetiva das Câmeras, dada $F$}.............................................. & 58\\
2 & {\bf RECONSTRUÇÃO DE UMA CÂMERA $P$}....................................... & 60\\
2.1 & {\bf Formulação do Problema}........................................................................ & 61
\end{tabular}
\end{flushleft}
\begin{center}
\begin{tabular}{llr}
2.2 & {\bf Sistema de Equações Polinomiais}.......................................................... & 65\\
3 & {\bf GEOMETRIA TRIFOCAL}.................................................................... & 66\\
3.1 & {\bf O Problema}............................................................................................... & 66\\
3.2 & {\bf Transferência Epipolar}............................................................................ & 68\\
3.3 & {\bf O Tensor Trifocal e as Relações de Incidência}................................... & 69\\
3.3.1 & \underline{Homografia Induzida por um Plano}............................................................. & 72\\
3.3.2 & \underline{Relações de Incidência entre Pontos e Retas}................................................ & 73\\
3.3.3 & \underline{Relação de Incidência para Retas Epipolares}............................................... & 74\\
3.4 & {\bf Transferências Usando o Tensor Trifocal}............................................ & 75\\
3.4.1 & \underline{Transferência de Pontos}................................................................................ & 75\\
3.4.2 & \underline{Transferência de Retas}................................................................................. & 77\\
3.5 & {\bf Reconstrução das Câmeras a partir do Tensor Trifocal}................... & 78\\
3.5.1 & \underline{Extração da Matriz Fundamental}................................................................ & 78\\
3.5.2 & \underline{Extraindo as Matrizes das Câmeras}............................................................. & 79\\
4 & {\bf GEOMETRIA TRIFOCAL CALIBRADA}........................................ & 81\\
4.1 & {\bf O Teorema 1: Homografia da Reta Epipolar}..................................... & 82\\
4.1.1 & \underline{Nova Abordagem para a Homografia da Reta Epipolar}.............................. & 83\\
4.2 & {\bf Dedução da IAC}...................................................................................... & 85\\
4.2.1 & \underline{Calibração e a Imagem da Cônica Absoluta}................................................ & 85\\
4.3 & {\bf O Teorema 2: Restrição Kruppa}.......................................................... & 87\\
4.4 & {\bf Mudança de Coordenadas Projetivas}.................................................. & 89\\
4.4.1 & \underline{Corregistro de Pontos Pertencentes a Diferentes Planos de Imagem}...........& 89\\
4.5 & {\bf O Teorema 3: Pontos e Epipolos Co-cônicos}..................................... & 90\\
4.6 & {\bf O Teorema 4: Interseção das Tangentes}............................................. & 91\\
4.7 & {\bf O Teorema 5: Projeção de $\omega$ em $B$}...................................................... & 91\\
4.7.1 & \underline{Representação Canônica de uma Cônica}..................................................... & 92\\
4.7.2 & \underline{Representação Canônica Paramétrica}.......................................................... & 96\\
4.7.3 & \underline{Demonstração do Teorema}.......................................................................... & 96\\
4.8 & {\bf O Teorema 6: Projeção de $\omega'$ em $B$}..................................................... & 98\\
4.9 & {\bf O Teorema 7: A Função que Relaciona os Epipolos}........................& 98\\
4.9.1 & \underline{Feixe de Cônicas}.......................................................................................... & 98\\
4.9.2 & \underline{Estabelecimento e Demonstração do Teorema}............................................ & 99\\
4.10 & {\bf O Teorema 8: A Curva $C$}...................................................................... & 101\\
4.11 & {\bf Teoremas 9 a 20: A Curva $G$ e suas Propriedades}........................... & 102\\
\end{tabular}
\end{center}

\begin{center}
\begin{tabular}{llr}
4.12 & {\bf Teoremas 21 a 25: Orientação e Convergência}................................. & 105\\
4.12.1 & \underline{A Restrição de Orientação}.......................................................................... & 105\\
4.12.2 & \underline{A Restrição de Convergência}...................................................................... & 105\\
4.13 & {\bf Incluindo a Terceira Imagem}............................................................... & 108\\
& {\bf CONCLUSÕES E TRABALHOS FUTUROS}.................................. & 110\\
& {\bf REFERÊNCIAS}..................................................................................... & 112\\
& {\bf APÊNDICE A} - Alguns Conceitos e Definições de Álgebra Linear........ & 115\\
& {\bf APÊNDICE B} - Alguns Conceitos e Definições de Geometria Algébrica & 118\\
\end{tabular}
\end{center} 