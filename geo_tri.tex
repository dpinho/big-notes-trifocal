\section{Geometria Trifocal}

\subsection{O Problema}

\subsection{Abordagem por Tensor Trifocal}

\subsection{Pesquisa Anteriore para Determinação de Câmera usando três Imagens}


\subsubsection{Abordagem de Schimd e Zisserman}

Este artigo lida com o problema de encontrar correpondências entre seguimentos de curvas quando temos diponíveis duas ou três imagens de câmeras deconhecidas. A matriz fundamental e o tensor trifocal são facilmente computados através de pontos de interesse mais o uso do RANSAC, mas os parâmetros intrínsecos não precisam necessariamente ser conhecidos. A abordagem clássica é descartar esses pontos de interesse e usar a geometria da câmera em estágios subsequentes de associação estéreo, para eliminar a ambiguidade entre pontos correspondentes ao longo da linha epipolar. Com duas imagens, o método padrão é usar a restrição epipolar para computar pontos correspondentes, o qual limita os canditados a estarem numa região estreita, e o uso de correleção cruzada normalizada, obtida a partir de fotometria, para restringir mais ainda nossas escolhas dos pontos. 

No artigo são analisados dois casos, binocular e trinocular, para responder a uma questão principal: existe alguma restrição adicional quando não utilizamos pontos mais sim pontos-tangentes ou geometria de curvas em geral disponíveis?\\

\noindent {\bf O Caso Binocular}

A busca por correspondências usando correlação fotométrica pode ser significantemente reduzida usando geometria de curvas (sem a utilização de contornos oclusos) numa pequena vizinhança. Considere uma aproximação de primeira ordem (planar) para a superfície de um objeto no qual aloja-se uma curva 3D. A correspondência entre as projeções desse plano em duas imagens são descritas por uma homografia de 8 parâmetros livres. É sabido que a matriz fundamental, que contém informações da geometria epipolar, fornece 5 restrições, o que reduz para 3 os graus de liberdade da homografia. Assim, o fato de que um ponto está associado a um outro ponto na reta epipolar já está sendo considerado nos cálculos. Dados alguns pontos correspondentes num fragmento de curva ao longo da linha epipolar, cada par de correspondência fornece duas restrições nas variáveis desconhecidas: como as curvas devem ser mapeadas umas nas outras, (i) a posição ao longo da linha epipolar define uma variável, já que o plano 3D é forçado a passar pelo ponto 3D reconstruído a partir dessa informação (a menos de uma homografia 3D). Assim, resta uma família de um parâmetro que pode ser otimizado usando o grau da correlação fotométrica. Dentre a família de um parâmetro de solução restante, os autores propuseram que, no lugar de otimizar através de correlação fotométrica, podemos apenas escolher o plano que coincide com o plano osculante da curva, o qual pode ser determinado usando a curvatura dos pontos correspondentes. Isto é apenas uma heurística, um método construtivo, onde não há garantia de que se trata da tangente à superfície, mesmo se a curva é plana. Contudo, empiricamente os autores expuseram que essa solução é boa o suficiente para criar uma fila ordenada das associações mesmo considerando os casos acima.

A questão de como conseguir um plano osculante (e a homografia associada a ele) foi a motivação para chegar a um resultado de interesse geral. Sabemos, através da posição e curvatura das extremidades correspondentes em duas imagens, que é possível estimar, unicamente, o plano osculante a uma curva se a calibração da imagem é completamente conhecida. Se apenas a matriz fundamental é conhecida, o plano osculante pode ser encontrado a menos da ambiguidade projetiva somente, mas a homografia associada a ele pode ser completamente determinada, o que é estudado nesse artigo.

\noindent {\bf O Caso Trinocular}

Os autores também propuseram um método para transferir a curvatura de duas imagens para uma terceira, dado um tensor trifocal (a calibração completa não é necessária). A interpretação da fórmula dada é usar o plano reconstruído a partir de duas imagens (conhecido a menos de uma ambiguidade projetiva 3D), e usá-lo para conseguir uma homografia relacionando as duas primeiras imagens com a terceira (sendo esta homografia independente dos parâmetros intrínsecos). Esta homografia define, então, a correspondência de pontos de quaisquer das duas primeiras imagens para a terceira. Faugeras e Robert foram os primeiros a proporem a transferência de curvatura a partir de duas imagens não calibradas para uma terceira, mas eles usaram par de matrizes fundamentais. O enriquecimento adicionado ao método de Faugeras proposto por Chimd e Zisserman é numericamente estável e imune a erros na interseção de linhas epipolares, que podem ser colineares ou aproximadamente colineares. O tensor trifocal manipula esses casos sem qualquer imprecisão.\\

\noindent {\bf Passo para um Sistema Prático} 

\begin{itemize}
\item Começar com a detecção de extremidades relacionadas em duas ou três imagens, e tentar associar fragmentos de curvas por inteiros.
\item Num primeiro estágio, considere apenas curvas alojadas no raio das linhas epipolares de cada uma das outras curvas, e têm tangências epipolares consistentes.
\item Num segundo estágio, tente encontrar fragmentos correspondentes de supostas curvas correspondentes através da integração do custo de extremidade-a-extremidade com possibilidade de associar fragmentos.
\item Usando apenas a geometria diferencial de curvas ganha-se restrições através de supostas extremidades associadas na terceira imagem. Isto funciona mesmo quando os parâmetros intrínsecos são desconhecidos, mas a ambiguidade permanece.
\item Restrições adicionais podem ser encontradas usando as aparências.
\item Para cada suposta associação, encontre um grupo de correlações no espaço gerado por um parâmetro se as curvaturas das associações são poucas (somente tangentes são utilizadas), ou não use buscas (planos osculantes) se as curvaturas acima são limites.
\item Um esquema das melhores escolhas dentre todas direciona para as associações finais.
\end{itemize}



