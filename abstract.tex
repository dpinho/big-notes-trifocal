\begin{center}
{\bf ABSTRACT}
\end{center}

\vspace{1 cm}

\begin{center}
\begin{minipage}{1\textwidth}
\setstretch{1}
\noindent PINHO, D. C. {\it Trifocal geometry in 3D reconstruction.} Dissertação (Mestrado em Modelagem Computacional) - Instituto Politécnico do Rio de Janeiro, Universidade do Estado do Rio de Janeiro, Nova Friburgo, 2016.
\end{minipage}
\end{center}

\vspace{1 cm}

\begin{center}
\begin{minipage}{1\textwidth}
\setstretch{1}
\qquad In this work we investigate some advantages of using trifocal geometry applied to 3D multiview reconstruction systems and multiple camera model estimation, as opposed to using pairwise epipolar geometry. We present a detailed exposition and interpretation of the most important mathematical tools in two key recent papers, aiming at tackling trifocal problems for general curvilinear structures in the future. While one of them shows the Hamilton's Quaternions and Gr\"obner Bases applications for computing the pose of one camera given 3D-2D corresponding structures, the other one is the most computationally efficient approach (to date) for trifocal reconstruction of camera systems. The first chapters and appendices present the basic theory about Projective Geometry, Linear Algebra and Algebraic Geometry useful for advancing research in the field of 3D multiview reconstruction and structure from motion systems.
\end{minipage}
\end{center}
 
\vspace{1 cm}

\begin{flushleft}
Keywords: Computer vision. 3D reconstruction. Trifocal geometry. Gr\"obner bases. 
\end{flushleft}

\newpage