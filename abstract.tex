\begin{center}
{\bf ABSTRACT}
\end{center}

\vspace{1 cm}

\begin{center}
\begin{minipage}{1\textwidth}
\setstretch{1}
\noindent PINHO, D. C. {\it Trifocal geometry in 3D reconstruction.} Dissertação (Mestrado em Modelagem Computacional) - Instituto Politécnico do Rio de Janeiro, Universidade do Estado do Rio de Janeiro, Nova Friburgo, 2016.
\end{minipage}
\end{center}

\vspace{1 cm}

\begin{center}
\begin{minipage}{1\textwidth}
\setstretch{1}
\qquad In this work we present some advantages using trifocal geometry applied to transfers and cameras computation, to confront using epipolar geometry in three view system. There are the detailed exposition of the most important mathematical tools in two papers. While one of them shows the Hamilton's Quaternions and Gr\"obner Bases applications to one camera's computation, the other one is the most efficient approach (at least until 2006) to trifocal system reconstruction of cameras. It's available in the first chapters and appendix, the basic theory about Projective Geometry, Linear Algebra and Algebraic Geometry useful to dissertation learnning.
\end{minipage}
\end{center}
 
\vspace{1 cm}

\begin{flushleft}
Keywords: Computer vision. 3D reconstruction. Trifocal geometry. Gr\"obner bases. 
\end{flushleft}

\newpage